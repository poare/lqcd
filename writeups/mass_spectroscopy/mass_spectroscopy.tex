\documentclass[11pt, oneside]{article}   	% use "amsart" instead of "article" for AMSLaTeX format
\usepackage[margin = 1in]{geometry}                		% See geometry.pdf to learn the layout options. There are lots.
\geometry{letterpaper}                   		% ... or a4paper or a5paper or ... 
%\geometry{landscape}                		% Activate for rotated page geometry
%\usepackage[parfill]{parskip}    		% Activate to begin paragraphs with an empty line rather than an indent
\usepackage{graphicx}				% Use pdf, png, jpg, or eps§ with pdflatex; use eps in DVI mode
								% TeX will automatically convert eps --> pdf in pdflatex		
\usepackage{amssymb}
\usepackage{amsmath}
\usepackage[shortlabels]{enumitem}
\usepackage{float}
\usepackage{tikz-cd}

\usepackage{amsthm}
\theoremstyle{definition}
\newtheorem{definition}{Definition}[section]
\newtheorem{theorem}{Theorem}[section]
\newtheorem{corollary}{Corollary}[theorem]
\newtheorem{lemma}[theorem]{Lemma}

\newcommand{\N}{\mathbb{N}}
\newcommand{\R}{\mathbb{R}}
\newcommand{\Z}{\mathbb{Z}}
\newcommand{\Q}{\mathbb{Q}}

%SetFonts

%SetFonts


\title{Mass Spectroscopy}
\author{Patrick Oare}
\date{}							% Activate to display a given date or no date

\begin{document}
\maketitle

\section{Overview}

The idea of mass spectrometry is to use the lattice to calculate two point functions, and then to take these 
two point functions and extract physics. The central equation to this is the following:
\begin{equation}
	\langle\mathcal O(\mathbf n, n_t)\overline{\mathcal O}(0)\rangle = \sum_k\langle 0|\hat{\mathcal O} 
	|k\rangle\langle k|\hat{\mathcal O}^\dagger |0\rangle\exp\left(-an_t E_k\right)~
	\label{eq:two_point}
\end{equation}
Here $|k\rangle$ represents any intermediate state which can be connected to the operator $\mathcal O$ 
with energy $E_k$. The operators $\hat{\mathcal O}$ and $\hat{\mathcal O}^\dagger$ are related to the 
objects $\mathcal O$ and $\overline{\mathcal O}$ by the following identity:
\begin{equation}
	\mathcal O(\alpha)\delta(\alpha - \alpha') := \langle \alpha' | \hat{\mathcal O} | \alpha\rangle
\end{equation}
where $|\alpha\rangle$ is a continuous eigenbasis of the space you are considering. 

Using the relation between two point functions and matrix elements, we may extract physics about the energies 
of nucleon states. To do this, we consider evaluating two point functions when $\mathcal O$ is an \textbf{interpolator} for the state we want to analyze. By an interpolator, we mean that $\mathcal O$ 
creates or destroys the state of interest. For example, a proton interpolator $\chi(n)$ will destroy a proton at 
site $n$, and its adjoint $\overline\chi(n)$ will create a proton at site n. 

So, suppose that we want to calculate the mass of a hadron $h$ with corresponding interpolator $\chi$. 
Then using Equation~\ref{eq:two_point}, we have:
\begin{equation}
	\langle\chi(\mathbf n, n_t)\overline\chi(0)\rangle = \sum_k\langle 0 |\hat\chi |k\rangle\langle k |
	\hat{\chi}^\dagger |0\rangle\exp(-an_t E_k)
\end{equation}
Because the operator $\hat{\chi}^\dagger$ creates a hadron state $\hat{\chi}^\dagger|0\rangle$, the 
sum over intermediate states $|k\rangle$ will only connect states with the quantum numbers of the hadron of 
interest to the ket $\hat{\chi}^\dagger|0\rangle$. So, the sum $\sum_k$ will only index over intermediate 
states which contain at least one hadron of species $h$. 

Consider taking the large $n_t$ limit of this equation. When $n_t$ gets large, intermediate states with a larger 
energy $E_k$ become exponentially suppressed, and the dominant term in this series is the state $|k\rangle$ 
with the lowest energy. Here is the key point: \textit{the lowest energy state is exactly the ground state, and the 
energy will be given (at zero momentum) by $E_0 = m_h$}. Thus, as we take $n_t$ to be larger and larger, 
we can extract the hadron mass $m_h$ by calculating the two point function of the interpolating operator:
\begin{equation}
	\langle\chi(n_t)\chi^\dagger(0)\rangle = A\exp(-a n_t E_0)(1 + O(e^{-an_t\Delta E}))
\end{equation}
where now $A$ is a constant we can determine with a data fit. 

\subsection{Momentum Projection}

\subsection{Effective Mass}

\subsection{Bootstrapping}

Assume now that we have calculated values of the interpolation function on $n$ gauge field configurations 
$\{U_i\}_{i = 1}^n$, and that our lattice has size $L^3\times T$. Let the values of the interpolators be 
contained in the set:
\begin{equation}
	\{C(i, t)\}_{i = 1, t = 1}^{n, T}
\end{equation}
where $C(i, t)$ is the value of the interpolator on the $i$th gauge field configuration, momentum projected 
to 0 on the $t$th time slice. Generally these values $C(i, t)$ will be stored in a matrix $C_{it}$ to make them 
easier to work with. If we simply want to ensemble average of these interpolators, at this point we could 
average them over the configuration index to find:
\begin{equation}
	\langle C(t)\rangle = \frac{1}{n}\sum_{i = 1}^n C(i, t)
\end{equation}
This would give us an average for our correlation function computed at each time slice, which is what we 
originally sought after.

However, this is not the best way to establish statistics on this problem. Instead, we consider a method 
called \textbf{bootstrapping}, which will allow us to produce a larger amount of ensembles from our 
initial data $C(i, t)$. With this method, we consider generating $N_{boot}$ new ensembles from our 
original one. To do this, fix $t$ and pick $n$ entries from $C(i, t)$ randomly \textit{with replacement}. 
Denote these $n$ samples by $U_1(i, t)$. Repeat this sampling procedure $N_{boot}$ times until 
you have a collection $U_1(i, t), ..., U_{N_{boot}}(i, t)$ of ensembles sampled from your original data. 
We may use the notation $U(b, i, t)$ for $U_b(i, t)$, where $b = 1, ..., N_{boot}$, $j = 1, ..., n$, and $t = 
1, ..., T$ to refer to these ensembles and to emphasize its structure as a 3-dimensional matrix. 

The ensemble $\{U_1, ..., U_{N_{boot}}\}$ will be used for our data analysis. To compute the correlation 
functions $\langle C(t)\rangle$, we will first determine the average over each ensemble:
\begin{equation}
	\langle C(t)\rangle_b := \frac{1}{n}\sum_{i = 1}^n U(b, i, t)
\end{equation}
The idea is that now we have an ensemble of average values for $\langle C(t)\rangle$, so we may compute 
statistics on the set $\{\langle C(t)\rangle_1, ..., \langle C(t)\rangle_{N_{boot}}$, i.e. we may take 
the average and (sample) variance of it as follows:
\begin{equation}
	\langle C(t)\rangle = \frac{1}{N_{boot}}\sum_{b = 1}^{N_{boot}}\langle C(t)\rangle_b
\end{equation}
\begin{equation}
	\sigma_{C(t)}^2 = \frac{1}{N_{boot} - 1}\sum_{b = 1}^{N_{boot}} (\langle C(t)\rangle_b - \langle C(t)
	\rangle_b)^2
\end{equation}

Using the bootstrap method, we may calculate statistics on more than just correlation functions. For example, 
to calculate the effective mass we can create $N_{boot}$ ensembles of effective mass values, then take the 
statistics on them. Doing this gives us a more accurate estimation of the mean and error on a data set.

\section{Example: Pion Mass}

\end{document}