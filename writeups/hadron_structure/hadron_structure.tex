\documentclass[11pt, oneside]{article}   	% use "amsart" instead of "article" for AMSLaTeX format
\usepackage[margin = 1in]{geometry}                		% See geometry.pdf to learn the layout options. There are lots.
\geometry{letterpaper}                   		% ... or a4paper or a5paper or ... 
%\geometry{landscape}                		% Activate for rotated page geometry
%\usepackage[parfill]{parskip}    		% Activate to begin paragraphs with an empty line rather than an indent
\usepackage{graphicx}				% Use pdf, png, jpg, or eps§ with pdflatex; use eps in DVI mode
								% TeX will automatically convert eps --> pdf in pdflatex		
\usepackage{amssymb}
\usepackage{amsmath}
\usepackage[shortlabels]{enumitem}
\usepackage{float}
\usepackage{tikz-cd}
\usepackage[compat=1.0.0]{tikz-feynman}   %note you need to compile this in LuaLaTeX for diagrams to render correctly
\usepackage{slashed}
\usepackage{simpler-wick}

\usepackage{amsthm}
\theoremstyle{definition}
\newtheorem{definition}{Definition}[section]
\newtheorem{theorem}{Theorem}[section]
\newtheorem{corollary}{Corollary}[theorem]
\newtheorem{lemma}[theorem]{Lemma}

\newcommand{\N}{\mathbb{N}}
\newcommand{\R}{\mathbb{R}}
\newcommand{\Z}{\mathbb{Z}}
\newcommand{\Q}{\mathbb{Q}}

%SetFonts

%SetFonts


\title{Hadron Structure}
\author{Patrick Oare}
\date{}							% Activate to display a given date or no date

\begin{document}
\maketitle

\section{Deep Inelastic Scattering (DIS)}

This section will begin by studying the Deep Inelastic Scattering (DIS) experiment, which is the process of a high energy electron scattering 
off of a proton via the exchange of a photon, as illustrated in Figure~\ref{fig:dis}. DIS is an example which will allow us to illustrate most of the 
interesting features of hadron structure, and we will be able to define many of our objects of interest with regards to this type of scattering 
specifically. 

\begin{figure}[H]
\centering
\feynmandiagram [vertical=a to b] {
  i1 [particle=\(e^{-}\)] -- [fermion, edge label'=\(p\)] a -- [fermion, edge label'=\(p'\)] i2 [particle=\(e^{-}\)],
  a -- [photon, edge label=\(q\)] b [blob],
  f1 [particle=\(X\)] -- [anti fermion] b -- [anti fermion, edge label'=\(P\)] f2 [particle=\(p^+\)],
};
\caption{The Deep Inelastic Scattering (DIS) diagram}~
\label{fig:dis}
\end{figure}
%\begin{figure}[H]
%\centering
%    \begin{tikzpicture}
%    \begin{feynman}
%    \vertex (a) {\(e^{-}\)};
%    \vertex [below right=of a] (b);
%    \vertex [above right=of b] (f1) {\(e^-\)};
%    \vertex [below=of b] (c);
%    \vertex [below left=of c] (f2) {\(p^+\)};
%    \vertex [below right=of c] (f3) {\(X\)};
%    \diagram* {
%    (a) -- [fermion] (b) -- [fermion] (f1),
%    (b) -- [boson, edge label'=\(q\)] (c),
%    (c) -- [anti fermion] (f2),
%    (c) -- [fermion] (f3),
%    };
%    \end{feynman}
%    \end{tikzpicture}
%\end{figure}

As the energy of the initial electron is increased, the transfer momentum $q$ becomes larger and larger until it is able 
to break up the proton into any particle that it can be broken into. As a result, the scattering starts off as elastic until it 
reaches some energy threshold, then becomes highly inelastic as the proton is smashed into different particles. Without 
any knowledge of the interaction that occurs at the bottom vertex, we can write down a surprising amount of information 
about the scattering process. The amplitude is:
\begin{equation}
	i\mathcal M = -ie\overline u(p')\gamma^\mu u(p) \frac{-ig_{\mu\nu}}{q^2}\hat{\mathcal M^\nu}(q) = 
	-\frac{e}{q^2}\overline u(p')\gamma^\mu u(p)\hat{\mathcal M}_\mu(q)
\end{equation}
where $i\hat{\mathcal M}(q)^\mu$ is the amplitude for the photon interacting with the proton and breaking it into a final 
state $|X\rangle$:
\begin{equation}
i\hat{\mathcal M}^\mu(q) =
\begin{gathered}
\feynmandiagram [small, vertical=a to b] {
  a [particle=\(\mu\)] -- [photon, edge label=\(q\)] b [blob],
  f1 [particle=\(X\)] -- [anti fermion] b -- [anti fermion] f2 [particle=\(p^+\)],
};
\end{gathered}~
\label{eq:dis_hadronic_diagram}
\end{equation}
If we want to calculate the unpolarized cross section, then we need the spin averaged matrix element:
\[
	\overline{|\mathcal M|^2} = \frac{1}{2}\sum_{spins, X}|\mathcal M|^2 = \frac{e^2}{2q^4}\left(\sum_{sr}\overline u_r(k')\gamma^\mu 
	u_s(k)\overline u_s(k)\gamma^\nu u_s(k')\right)\left(\sum_{spins}\mathcal M_\mu(q)\mathcal M_\nu^*(q)\right)
\]
\begin{equation}
	= \frac{e^2}{2q^4} tr\left[\slashed{k'}\gamma^\mu \slashed{k}\gamma^\nu\right] \left(\sum_{spins}\mathcal M_\mu(q)\mathcal M_\nu^*(q)\right)~
	\label{eq:dis_spin_averaged}
\end{equation}
We can integrate this over phase space to get a differential cross section in the lab frame, which we use to define two tensors:
\begin{equation}
	\left(\frac{d\sigma}{d\Omega dE'}\right)_{lab} = \frac{\alpha_e^2}{4\pi m_p q^4}L^{\mu\nu} W_{\mu\nu}
\end{equation}

The first tensor $L{\mu\nu}$ is the \textbf{leptonic tensor} and describes all aspects of the scattering related to the scattering 
of the initial and final electrons, and how they interact with the photon. Despite not knowing much about the scattering process 
$\gamma^* p^+\rightarrow X$, we can still explicitly write this tensor down by working through the math starting from 
Equation~\ref{eq:dis_spin_averaged}:
\begin{equation}
	L^{\mu\nu} := \frac{1}{2}tr\left[\slashed{k'}\gamma^\mu\slashed{k}\gamma^\nu\right] = 2(k'^\mu k^\nu + k'^\nu k^\mu - k\cdot k' g^{\mu\nu})
\end{equation}

The second tensor $W_{\mu\nu}$ is the \textbf{hadronic tensor}, and describes the physics in the hadronic part of the process, 
namely when the proton interacts with the mediating photon. Although we do not know many details about the process, we can still 
use general principles of symmetry to make some headway into describing the physics, without explicitly knowing information 
about the final state or the interaction. Explicitly, the hadronic tensor is:
\begin{align}
	e^2\epsilon_\mu\epsilon_\nu^* W^{\mu\nu} :=& \frac{1}{2}\sum_{X, spins} \int d\Pi_X (2\pi)^2\delta^4\left(\sum p\right)
	|\mathcal M(\gamma^*p^+\rightarrow X)|^2~
	\label{eq:hadronic_tensor}~
	\\
	& = 2m_{p^+}|\vec q|\sum_X \sigma(\gamma^*p^+\rightarrow X)~
	\label{eq:hadronic_tensor_cross}
\end{align}
and depends explicitly on the amplitude for the virtual proton to scatter of the proton and produce the state $|X\rangle$, where $X$ 
is any possible product state that can be produced by this reaction. Notice that the hadronic tensor is proportional to a sum of cross 
sections $\sigma(\gamma^* p^+\rightarrow X)$-- this fact will be important later. Now, we will use this equation and this 
decomposition to explore the physics of this problem, and will show that we can predict a surprising amount without knowing 
the details of the interaction in Equation~\ref{eq:dis_hadronic_diagram}.

\subsection{Form Factors}

We can exploit the symmetry of the problem to heavily constrain the form of the hadronic tensor, and 
extract physics from this constrained form. This is the \textbf{method of form factors}. We know the following things about 
$W^{\mu\nu}$:
\begin{enumerate}
	\item It may only depend on the momentum $q^\mu$ and $P^\mu$, since the external momenta in the state $|X\rangle$ are 
	integrated over in the phase space integral. 
	\item It must be symmetric, i.e. $W^{\mu\nu} = W^{\nu\mu}$, because we assume the initial photon is unpolarized. 
	\item It must obey the Ward identity applied to the diagram in Equation~\ref{eq:dis_hadronic_diagram}, so $q_\mu W^{\mu\nu} = 0$. 
\end{enumerate}

Items 1 and 2 on the list say that we can obtain a general form for $W^{\mu\nu}$ by forming all symmetric rank 2 combinations 
of the Lorentz vectors $q^\mu$ and $P^\mu$, as well as the metric $g^{\mu\nu}$, and examining all linear combinations of 
them. This gives 3 parameters we can vary in the linear combination for the general form for $W^{\mu\nu}$. 
However, the Ward identity $q_\mu W^{\mu\nu} = 0$ constrains the form these combinations can take on, and so we actually 
only have 2 coefficients we can vary in the linear combination, which we will call $W_1$ and $W_2$. These coefficients are called 
\textbf{form factors}, and we can explicitly write the most general form of $W^{\mu\nu}$ consistent with our constraints:
\begin{equation}
	W^{\mu\nu} = \left(-g^{\mu\nu} + \frac{q^\mu q^\nu}{q^2}\right) W_1 + \left(P^\mu - \frac{P\cdot q}{q^2}q^\mu\right)\left(P^\nu - 
	\frac{P\cdot q}{q^2}q^\nu\right) W_2~
	\label{eq:form_factors}
\end{equation}

The form factors $W_1$ and $W_2$ are Lorentz scalars, and therefore must be functions of the Lorentz scalars we can generate with 
our dynamical variables $q^\mu$ and $P^\mu$. The three combinations that we can create are $q^2$, $P^2$, and $P\cdot q$. Note that 
$P^2 = m_p^2$ is not a dynamical variable because we assume the initial photon is on shell, but since we do not make this assumption 
with the photon, $q^2$ is a dynamical variable that our form factors can depend on. Thus we can write the form factors as functions 
of $q^2$ and $P\cdot q$. 

To massage this into a nicer form, let $Q := \sqrt{-q^2}$ be the energy scale of the collision, and define the dimensionless ratio:
\begin{equation}
	x := \frac{Q^2}{2 P\cdot q}
\end{equation}
This is called the \textbf{Bjorken variable}, and is an important variable in describing hadron structure. Physically, you can think of $x$ as a 
momentum fraction. So, we will consider our form factors as functions of $x$ and $Q$:
\begin{equation}
	W_1 = W_1(x, Q)\;\;\;\;\;\;\;\;\;\;\;\;\;\;\;\;\;\;\;\; W_2 = W_2(x, Q)
\end{equation}

Using the general form Equation~\ref{eq:form_factors} of the hadronic tensor, we can plug plug this into our cross section to express 
it explicitly in terms of $W_1$ and $W_2$:
\begin{equation}
	\left(\frac{d\sigma}{d\Omega dE'}\right)_{lab} = \frac{\alpha_e^2}{8\pi E^2\sin^4(\theta / 2)}\left(\frac{m_p}{2} W_2(x, Q)\cos^2\frac{\theta}{2} + 
	\frac{1}{m_p} W_1(x, Q)\sin^2\frac{\theta}{2}\right)~
	\label{eq:sigma_ff}
\end{equation}
This should be pretty remarkable: without knowing anything about the final state $|X\rangle$ or really any details of the hadronic process 
described by the diagram of Equation~\ref{eq:dis_hadronic_diagram}, we have expanded an observable in terms of the two form 
factors $W_1$ and $W_2$. This means that the form factors can be experimentally measured, and then we can use those measurements 
to make other predictions about the process. 

\subsection{The Parton Model}

A \textbf{parton} is a point-like particle which has no composite substructure-- examples of partons are the electron, neutrino, and quarks. 
When nuclear structure was being studied, the proton was originally believed to be a parton as well until experiments like DIS showed 
that it instead has a substructure of quarks and gluons. 

Feynman coined the \textbf{parton model} of the proton, in which he assumed the proton was made up of constituent 
partons that interact weakly and are almost free particles. Suppose that we assume the proton is made up of partons of mass 
$\{m_i\}_{i}$. In DIS, we can assume the photon interacts with a single parton, say parton $j$. Then we can describe this process 
with the diagram in Figure~\ref{fig:dis} where we replace the proton line with a parton line with incoming momentum $p_j$ and external 
momentum $p_j'$. Momentum conservation gives us $p_j + q = p_j'$, and squaring both sides we get:
\begin{equation}
	\frac{Q^2}{2p_j\cdot q} = 1
\end{equation}
Since the parton is a constituent of the proton with momentum $P$, assume that it has a fraction $\xi$ of the proton's 
momentum $P$, i.e. $p_j = \xi P$. Then using the equation above, we find the Bjorken variable is:
\begin{equation}
	x = \frac{Q^2}{2P\cdot q} = \xi \frac{Q^2}{2p_j\cdot q} = \xi
\end{equation}
so with these assumptions, the Bjorken $x$ is exactly the momentum fraction of the parton which is involved in the scattering 
process. 

Note that in the physical world described fully by QCD, the Bjorken variable is not the momentum fraction of the 
parton. This will be approximately true, \textbf{but in general $x$ and $\xi$ are different, although they are related}. Thinking of the physical 
interpretation of $x$ as the momentum fraction is helpful, but not exactly accurate. 

In the actual proton, partons are not free but interact. Assuming they interact via the electromagnetic force, we can 
examine the process $e^-q\rightarrow e^-q$ through a photon exchange, where $q$ is the parton. When we studied IR divergences, 
we calculated this for the $e^-e^+\rightarrow\mu^-\mu^+$ scattering, and found that the form factor $F_1(q^2)$ ran as $F_1(q^2)\propto 
log(Q^2)$ and $log^2(Q^2)$ when the initial momentum was fixed (i.e. when we vary $x$). This weak running of the form factors 
as we vary $Q$ at fixed $x$ is called \textbf{Bjorken scaling} and applies in the parton model as well: when we work at fixed $x$: 
the form factors stay relatively constant as we vary the energy scale of the collision. 

Another key ingredient of the parton model is an object known as a \textbf{Parton Distribution Function (PDF)}. PDFs are functions 
$f_i(\xi)d\xi$ which give the probability density for scattering off of parton $i$ having momentum fraction $\xi$. Since the Bjorken 
variable has an interpretation as the momentum fraction of the proton which is contained by the parton, we will write out PDFs as 
$f_i(x)dx$. The parton model assumes that we can factorize the DIS cross section into partonic ones:
\begin{equation}
	\sigma(e^- P^+\rightarrow e^- X) = \sum_i\int_0^1dx f_i(x)\hat{\sigma}(e^- p_i\rightarrow e^- X)~
	\label{eq:sigma_pdf}
\end{equation}
Here the sum runs over all partons $p_i$ contained in the proton, and $\hat\sigma$ is the cross section for the partonic process. 

Now, we assume that the partons only interact with the proton through QED and have charges $Q_i$ ($Q_i$ is the electric charge of 
different types of quarks). This allows us to use the Rosenbluth formula for spin 1/2 QED scattering with form factors $F_1\equiv 1$ and 
$F_2\equiv 0$. Plugging these into Equation~\ref{eq:sigma_pdf} and turning it into a differential cross section, we can compare it to 
Equation~\ref{eq:sigma_ff} to read off the form factors $W_1$ and $W_2$:
\begin{equation}
	W_1(x, Q) = 2\pi\sum_i Q_i^2 f_i(x) \hspace{3cm}W_2(x, Q) = \frac{8\pi x^2}{Q^2}\sum_i Q_i^2 f_i(x)~
	\label{eq:ff_decomp}
\end{equation}
The relation between $W_1$ and $W_2$ is called the \textbf{Callan-Gross relation}, and was used as experimental evidence for the 
parton model. It is a direct consequence of the fact that quarks have spin 1/2. The relation is:
\begin{equation}
	W_1(x, Q) = \frac{Q^2}{4x^2} W_2(x, Q)
\end{equation}

Equation~\ref{eq:ff_decomp} allow us to write out the proton form factors in terms of PDFs (note here we adopt the notation that the pdf $q(x) := 
f_q(x)$, so for example we use $u(x)$ to denote $f_u(x)$). We can see that the proton has $W_1$ form factor:
\begin{equation}
	W_1(x, Q) = 2\pi\left(\frac{4}{9} u + \frac{4}{9}\bar u + \frac{1}{9} d + \frac{1}{9}\bar d+ \frac{4}{9} s + \frac{4}{9}\bar s + ...\right)
\end{equation}
This can then be used to determine the form factors of the neutron, using isospin symmetry $u\leftrightarrow d$. 

PDFs must satisfy certain constraints to be valid probability distributions: namely, they must satisfy certain \textbf{sum rules} related to 
particle number conservation. As an example of this, consider the up quarks in the proton. While at any given time there are 2 valence quarks, 
we can also have $q\overline q$ production to create sea quarks. Since the total number of up quarks is conserved ($N_u + N_{\bar u} = 2$) 
we must have:
\begin{equation}
	\int_0^1dx \left(f_{u}(x) - f_{\bar u}(x)\right) = 2
\end{equation}
because for a single quark PDF, the total integral $\int_0^1dx q(x) = N_q$. Similar rules hold in the proton for down ($N_d = 1$) and other 
flavor ($N_f = 0$ for $f = s, c, b, t$) quarks. In general, a sum rule will apply anytime there is a conserved particle number. The proton is a bound 
state with specific quantum numbers for its number of (each flavor of) quarks, which implies the sum rules for the quark PDFs. Since specifying a 
bound hadron or meson state also specifies its quantum numbers, this means that \textit{each bound QCD state will come with its own set of sum 
rules} to reflect its general make-up. Note because gluon number is not conserved, there is \textit{no general sum rule for the gluon PDFs in a QCD 
bound state}. 

Finally, there is an additional sum rule for momentum conservation:
\begin{equation}
	\sum_i \int_0^1 dx\, x f_i(x) = 1
\end{equation}

\newpage
\section{Renormalization of PDFs: The DGLAP Equations}

Bjorken scaling holds quite well in actual experiments, but because the parton model is not a completely accurate physical theory (it does not 
allow for interactions between the constituent partons which are found in actual QCD), Bjorken scaling is violated to some extent. Examining the 
amount to which Bjorken scaling is violated will lead us to the DGLAP equations, which describe how parton PDFs mix under renormalization. 
To study this, define a modified hadronic tensor $\hat{W}_{\mu\nu}$, which is the definition in Equation~\ref{eq:hadronic_tensor} with $p^+$ 
replaced with a parton $p_i$. Similarly, let $z_i := Q^2 / 2p_i\cdot q$ be the counterpart of the Bjorken variable for a specific parton. Because 
$p_i = \xi P$ with $P$ the total momentum of the incoming proton, we see that $x = \xi z_i$. 

We can use this information to relate $W_{\mu\nu}$ to $\hat W_{\mu\nu}$. We can decompose the diagram defining $W_{\mu\nu}$ into 
a sum of diagrams where the photon interacts with each parton, i.e. we sum over all partons. We must consider the partons carrying an 
arbitrary momentum fraction of the proton constrained by $x = \xi z_i$, so we integrate over $d\xi f_i(\xi)$ and $dz_i$ with an additional 
$\delta$ function to enforce that $x = \xi z_i$. Note whenever we integrate over $\xi$, we must integrate $d\xi f_i(\xi)$ because $f_i(\xi)$ 
is the density that describes how the momentum of the parton is distributed. This amounts to:
\begin{equation}
	W_{\mu\nu}(x, Q) = \sum_i\int_0^1 dz\int_0^1 d\xi f_i(\xi) \hat{W}_{\mu\nu}(z, Q)\delta(x - \xi z) = \sum_i\int_x^1 \frac{d\xi}{\xi} f_i(\xi)
	\hat{W}_{\mu\nu}\left(\frac{x}{\xi}, Q\right)~
	\label{eq:parton_tensor}
\end{equation}
We next take the trace of the hadronic tensor, and use that to define the form factor $W_0$ (with an identical definition for $\hat{W_0}(z, Q)$):
\begin{equation}
	W_0(x, Q) := -g_{\mu\nu}W^{\mu\nu} = 3W_1(x, Q) - \left(m_p^2 + \frac{Q^2}{4x^2}\right)W_2(x, Q)\xrightarrow{Q >> m_p} 3W_1 
	- \frac{Q^2}{4x^2}W_2
\end{equation}
Plugging in the Callan-Gross relation, we see that for large $Q / x$ we have $W_0 = 2W_1$ to first order, and using Equation~\ref{eq:ff_decomp} 
allows us to relate $W_0$ directly to the PDFs:
\begin{equation}
	W_0(x, Q) = 4\pi\sum_i Q_i^2 f_i(x)~
	\label{eq:ff_to_pdf}
\end{equation}
Since $W_0$ is easily calculated in perturbation theory (it is essentially just the sum of unpolarized cross sections $\gamma^* P\rightarrow X$), we 
will use this \textit{to define the PDF $f_i(x)$}. In particular, we can compute $W_0$ to higher orders in perturbation theory to determine 
the PDFs $f_i(z)$. Explicitly, we will compute the next to leading order contributions to the cross section $\gamma^* \rho\rightarrow \rho$, where 
$\rho$ is a parton, and define this to be $\hat W_0^{NLO}$. 

This calculation will be very similar to the computation of IR divergences in the $e^+e^-\rightarrow\mu^+\mu^-$ cross section. Recall in that 
case we had to compute real emission graphs in addition to the regular cross section to cancel the IR divergences, and when we summed 
the loop diagrams with the emission graphs to each order in perturbation theory, the divergences vanished. We must do this in the same 
way for this computation. 

The next to leading order correction to $W_0$ can be calculated by considering the following amplitudes. First, the tree level vertex:
\begin{equation}
    i\mathcal M_0 =
    \begin{gathered}
        \feynmandiagram [small, vertical=a to b] {
                a [particle=\(\gamma^*\)] -- [photon, edge label=\(q\)] b,
                	f1 [particle=\(\rho\)] -- [anti fermion, edge label' = \(p_f\)] b,
		b -- [anti fermion, edge label' = \(p_i\)] f2 [particle=\(\rho\)],
        };
\end{gathered}~
\label{eq:tree_vertex}
\end{equation}
Next, the vertex correction from an incoming parton:
\begin{equation}
    i\mathcal M_V =
    \begin{gathered}
        \feynmandiagram [vertical=a to b] {
                a [particle=\(\gamma^*\)] -- [photon, edge label=\(q\)] b,
                c -- [anti fermion] b -- [anti fermion] d -- [gluon, half right] c,
                	f1 [particle=\(\rho\)] -- [anti fermion, edge label' = \(p_f\)] c,
		d -- [anti fermion, edge label' = \(p_i\)] f2 [particle=\(\rho\)],
        };
\end{gathered}~
\label{eq:virtual_emission}
\end{equation}
The next to leading order contribution in $W_0$ will go as $\mathcal M_0\mathcal M_V^* + h.c.$, i.e. will be the interference terms between 
this diagram and the original vertex. We will call this contribution $\hat W_0^V$. The second contribution to $\hat W_0^{NLO}$ is the real emission 
graphs from tree level parton scattering:
\begin{equation}
    i\mathcal M_R =
    \begin{gathered}
        \feynmandiagram [horizontal=a to b] {
		i2 -- [fermion, edge label' = \(p_i\)] a,
        		i1 [particle=\(\gamma^*\)] -- [photon, edge label' = \(q\)] a,
		a -- [fermion] b,
		b -- [fermion, edge label' = \(p_f\)] f1 [particle = \(\rho\)],
		b -- [gluon, edge label' = \(p_g\)] f2,
        };
\end{gathered}
+
\begin{gathered}
        \feynmandiagram [vertical=a to b] {
        		i1 [particle=\(\gamma^*\)] -- [photon, edge label' = \(q\)] a,
		b -- [gluon, edge label' = \(p_g\)] f2,
		b -- [anti fermion, edge label' = \(p_i\)] f1 [particle = \(\rho\)],
		i2 -- [anti fermion, edge label' = \(p_f\)] a,
		a -- [anti fermion] b,
        };
\end{gathered}
~
\label{eq:real_emission}
\end{equation}
The cross section from considering $\mathcal M_R\mathcal M_R^*$ in this process is the same order as $\hat W_0^V$, and so will contribute 
to $\hat W_0^{NLO}$ as well. We will call this contribution $\hat W_0^R$. So, the total next to leading order correction to the form factor $W_0$ is:
\begin{align}
	\hat W_0(z, Q) &= \hat W_0^{LO} + \hat W_0^V + \hat W_0^R \\
	&= 4\pi Q_i^2\left[\left(\delta(1 - z) - \frac{1}{\epsilon}\frac{\alpha_s}{\pi} P_{qq}(z)\left(\frac{4\pi\mu^2}{Q^2}\right)^{\frac{\epsilon}{2}}
	\frac{\Gamma(1 - \epsilon / 2)}{\Gamma(1 - \epsilon)}\right) + \textnormal{finite terms}\right]
\end{align}
The important piece of this cross section is the $P_{qq}(z)$ function. It is known as the \textbf{DGLAP Splitting Function}, and will serve a key 
role in the renormalization of our PDFs. It is defined as:
\begin{equation}
	P_{qq}(z) := C_F\left[(1 + z^2)\left[\frac{1}{1 - z}\right]_+ + \frac{3}{2}\delta(1 - z)\right]
\end{equation}
where $\left[\frac{1}{1 - z}\right]_+$ is the distribution defined under the integral by:
\begin{equation}
	\int_0^1 dz\frac{f(z)}{[1 - z]_+} := \int_0^1 dz\frac{f(z) - f(0)}{1 - z}
\end{equation}
Note that this definition of the distribution forces the splitting function to integrate out to 0:
\begin{equation}
	\int_0^1 dz\, P_{qq}(z) = 0~
	\label{eq:splitting_integral}
\end{equation}

We can plug our expression for $\hat{W_0}(z, Q)$ into Equation~\ref{eq:parton_tensor} to determine the NLO expression for $W_0(x, Q)$:
\begin{align}
	W_0(x, Q) &= \sum_i \int_x^1\frac{d\xi}{\xi} f_i(\xi)\hat{W}_0\left(\frac{x}{\xi}, Q\right)  \\
	&= 4\pi\sum_i Q_i^2\int_0^1\frac{d\xi}{\xi} f_i(\xi)\left[\delta\left(1 - \frac{x}{\xi}\right) - \frac{\alpha_s}{2\pi} P_{qq}\left(\frac{x}{\xi}\right)
	\left(\frac{2}{\epsilon} + \log\frac{\tilde\mu^2}{Q^2}\right)
	 + \textnormal{finite terms}
	\right]
\end{align}
where $\tilde\mu^2 := 4\pi e^{-\gamma}\mu^2$ (with $\gamma$ the Euler-Mascheroni constant). This expression comes from expanding out some of 
the terms in small $\epsilon$ using:
\begin{equation}
	\left(\frac{4\pi\mu^2}{Q^2}\right)^{\frac{\epsilon}{2}}\frac{\Gamma(1 - \epsilon / 2)}{\Gamma(1 - \epsilon)} = 1 + \log\left(\frac{4\pi e^{-\gamma}
	\mu^2}{Q^2}\right)\frac{\epsilon}{2} + O(\epsilon^2)
\end{equation}

Equation~\ref{eq:splitting_integral} implies that although there is a $\frac{1}{\epsilon}$ pole int the expression, when we integrate $W_0(x, Q)$ over 
$x$ to get the total cross section, the pole vanishes. Because the total cross section is the physical observable, this means that we have evaluated 
the process correctly. 

We can subtract off the value of $W_0(x, Q_0)$ at some reference point as our renormalization condition to get a finite observable answer for 
our form factor:
\begin{equation}
	\Delta W_0(x, Q, Q_0) = 2\alpha_s \sum_i Q_i^2\int_x^1 \frac{d\xi}{\xi}f_i(\xi)P_{qq}\left(\frac{x}{\xi}\right)\log\frac{Q^2}{Q_0^2}~
	\label{eq:deltaW}
\end{equation}
At this point, we have enough tools to determine the renormalization group equation for $f_i(x)$. At any arbitrary scale $\mu$, we assert that the 
PDFs that we measure are a derived quantity from $W_0$, as we discussed earlier after Equation~\ref{eq:ff_to_pdf}. This allows us to 
determine the PDFs at any energy scale $\mu$ as:
\begin{equation}
	W_0(x, Q = \mu) = 4\pi\sum_i Q_i^2 f_i(x, \mu)
\end{equation}
Taking a finite difference of this equation and matching it to Equation~\ref{eq:deltaW}, we find:
\begin{align}
	\Delta W_0(x, \mu_1, \mu_0) &= 4\pi\sum_i Q_i^2 (f_i(x, \mu_1) - f_i(x, \mu_0)) \\
	&= 2\alpha_s \sum_i Q_i^2\int_x^1 \frac{d\xi}{\xi}P_{qq}\left(\frac{x}{\xi}\right)
	\left[f_i(x, \mu_1)\log\mu_1^2 - f_i(x, \mu_0)\log\mu_0^2\right]
\end{align}
For small $\delta\mu = \mu_1 - \mu_0$, this gives us:
\begin{equation}
	f_i(x, \mu_1) - f_i(x, \mu_0) = \frac{\alpha_s}{2\pi}\int_x^1\frac{d\xi}{\xi} P_{qq}\left(\frac{x}{\xi}\right) f_i(x, \mu_0)\log\frac{\mu_1^2}{\mu_0^2}
\end{equation}
Taking $\delta\mu$ to be infinitesimal, we get a renormalization group equation for the PDFs $f_i(x, \mu)$. This is known as the \textbf{DGLAP 
evolution equation}, and governs the scale evolution of the PDFs:
\begin{equation}
	\mu\frac{d}{d\mu} f_i(x, \mu) = \frac{\alpha_s}{\mu}\int_x^1\frac{d\xi}{\xi} f_i(\xi, \mu) P_{qq}\left(\frac{x}{\xi}\right)
\end{equation}

This equation is very important as we consider renormalization of our PDFs, but the theory is not yet complete. In hadrons like the proton, we can 
actually get \textit{mixing between different PDFs as we flow to different energy scales}. This is because we did not perform the full 
computation for the NLO contributions to the cross section, but only considered the case when the initial partons were quarks. When we defined 
the hadronic tensor in Equation~\ref{eq:hadronic_tensor_cross} via a sum of cross sections and factored it through the partons contained in 
the proton, we must be careful to consider \textit{all} the partons contained in the proton. The contribution to $\hat{W}_0^{NLO}$ that we considered 
is from the partonic reaction $\gamma^* q\rightarrow X$ ($q$ is a quark), where in our case $X = q$ or $X = qg$. However, the proton also 
contains gluons and antiquarks-- these are partons just as quarks are, so to calculate the full $\hat{W}_0^{NLO}$ we must also compute cross 
sections with initial state gluons and antiquarks, like $\sigma(\gamma^* g\rightarrow q\bar q)$. 

Upon doing these calculations, one will find that the RGEs describing gluon PDF and quark PDF evolution are in fact coupled, meaning 
that upon RG flow, gluon PDFs and quark PDFs mix. These RGEs can be written in the following form:
\begin{equation}
	\mu\frac{d}{d\mu}\begin{pmatrix} f_i(x, \mu) \\ f_g(x, \mu)\end{pmatrix}
	= 
	\sum_j\frac{\alpha_s}{\pi}\int_x^1\frac{d\xi}{\xi}
	\begin{pmatrix}
		P_{q_i q_j}\left(\frac{x}{\xi}\right) & P_{q_i g}\left(\frac{x}{\xi}\right) \\
		P_{g q_j}\left(\frac{x}{\xi}\right) & P_{g g}\left(\frac{x}{\xi}\right) \\
	\end{pmatrix}
	\begin{pmatrix}
		f_j(\xi, \mu) \\ f_g(\xi, \mu)
	\end{pmatrix}
\end{equation}
where $f_i(x, \mu)$ is the PDF for the quark $q_i$ and $f_g(x, \mu)$ is the gluonic PDF, and the $P_{IJ}$ are splitting functions which can be 
computed in perturbation theory much like $P_{qq}$ above. This mixing of PDFs is very important practically for 
calculations. It means that although we can fix the PDFs at some scale with a renormalization condition to be independent, as soon as we 
begin flowing to different energy scales, the PDFs will begin to mix and we cannot fully understand the evolution of one PDF by itself. 
We thus require knowing all the PDFs for a system to describe its RG evolution fully, which requires more work. This mixing should also make 
sense intuitively. The quantum numbers and symmetries of QCD allow gluons and quarks to be interchanged and do not force a reason for them to 
be separate, so because there is no reason enforced by symmetries to keep the PDFs separate under renormalization. 

\newpage
\section{General Theory}

\subsection{Form Factors}

As we saw in the previous section, form factors are objects which allow us to probe the structure of composite particles. Formally, 
they are simply expansion coefficients in a given basis of Lorentz vectors, which can be given physical interpretation by 
examining their structure in various limits. 

For another example of this, consider the general QED vertex function $\Gamma^\mu$, which includes radiative corrections. 

\subsection{Parton Distribution Functions (PDFs)}

\subsection{Generalized Parton Distributions (GPDs)}

\newpage
\section{Computational Methods}

In all of the following sections, quark fields will be denoted by:
\begin{equation}
	q_\alpha^a(x)
\end{equation}
where Greek indices like $\alpha$ are indices in Dirac space and English indices like $a$ are in color space. Lorentz indices will also be 
Greek, and generally be obviously different than Dirac indices from their use. $q$ represents the flavor of the quark, so for example $u_1^1(x)$ 
would be the field representing the spin up component of the up quark with color $1$.

\subsection{Interpolating Operators}

Interpolating operators are used in our calculations to create states with fixed quantum numbers-- they are analogous to creation and annihilation 
operators, but for more complex states which can be bound states of multiple quarks. We will denote such operators by $\chi$-- suppose 
$\chi$ is an interpolating operator for a particle $P$. Then $\chi$ will annihilate a particle of type $P$ and $\bar\chi$ will create a particle of type 
$P$, and so $\bar\chi |\Omega$ will be a state containing a single particle of type $P$. 

We first discuss meson interpolators, then move onto hadron interpolators. \textbf{Mesons} are QCD bound states which are formed from two 
quarks. Like any bound state, they are color singlets, and the composition of their interpolating operators are a direct result of the quantum numbers 
of the state. 

For example, consider the pions $\pi^\pm$. These have isospin\footnote{A brief note on isospin. This is a quantum number of quarks 
which allows $u$ and $d$ quarks to be rotated into one another under the fundamental representation of $SU(2)$. As such, they both have isospin 
charge $I = \frac{1}{2}$, but up quarks have $I_z = +\frac{1}{2}$ while down quarks have $I_z = -\frac{1}{2}$. This means we can write them in a 
doublet $\begin{pmatrix} u \\ d \end{pmatrix}$ in the $I_z\propto\sigma_z$ eigenbasis.} $I = 1$ and $I_z = \pm 1$, charge $Q = \pm e$, spin 
$J = 0$, and parity $P = -1$. Because the pions have isospin $1$, the quarks they are composed of must be up or down quarks. To get a charge 
of $+e$, the $\pi^+$ must be composed of a up and antidown and the $\pi^-$ must be composed of a down and antiup. To satisfy parity, we must 
sandwich a $\gamma_5$ between the individual quark fields. So, we can postulate that the pion interpolator is:
\begin{align}
	\chi_{\pi^+}(x) &= \bar d(x)\gamma_5 u(x) \\
	\chi_{\pi^-}(x) &= \bar u(x)\gamma_5 d(x)
\end{align}

TODO add in transformation law verifications.

The general meson interpolator has a similar form. It is a linear combination of quark field bilinears with a Dirac matrix put in between the quark fields 
to make the interpolator have the correct transformation laws:
\begin{equation}
	\chi_M = \sum_{ij} \bar q_{i} \Gamma \bar q_j = \sum_{ij} \bar q_{i\alpha}^a \Gamma_{\alpha\beta} q_{j\beta}^a
\end{equation}
Note here $q_i$ and $q_j$ are different flavors of quarks, and we are still contracting each of the Dirac indices and the color matrices together, as 
in the expanded formula. 

\textbf{Baryons} are bound states of QCD made out of three quarks. As such, they get much more complicated. Whereas in the meson interpolators 
we could just contract the color indices together with the identity to form a singlet, now we must contract each of the three fields with a rank 
3 tensor to form a singlet. The obvious way to do this is to contract the color indices with $\epsilon^{abc}$. We will also have a free Dirac index 
on our interpolator, which can be used to create and destroy baryons with a given polarization in Dirac space. For example, the proton interpolator 
is:
\begin{equation}
	\chi^p_\alpha = \epsilon^{abc} (u^{T a}(x) C\gamma_5 d^b(x)) u_\alpha^c(x)
\end{equation}
where $T$ is a transpose on the Dirac indices. 

TODO Question: Why do we use a transpose in the proton interpolator instead of just Dirac conjugating it? Don't we want the thing multiplying 
$u_\alpha^c(x)$ to be a Dirac singlet? 

TODO Question: When does the interpolating operator have Dirac indices? Pion interpolator doesn't have Dirac index but proton does-- in general, is 
this because the pion is a meson and the proton is a baryon? Seems like mesons interpolators are all of the form $\bar q \Gamma q$, while 
baryon interpolators are more complicated. It's not a spin thing, because some mesons have spin 1 and still appear to be Dirac singlets. 

\subsection{Wick's Theorem}

TODO write on how to use it. Include example from Equation~\ref{eq:pion_threept}

\subsection{Three Point Functions}

Three point functions are more difficult to calculate on the lattice than two point functions. However, these will ultimately be the missing 
pieces that we need to calculate many types of matrix elements on the lattice, which we will do in the next section. We will be computing 
momentum projected 3-point functions. These Green's functions correspond to the amplitude to to create a particle at the origin at time 0, 
insert a current to interact with this particle at time $\tau$ and with momentum $\vec p$, and then annihilate the particle with momentum 
$\vec p'$ at time $t$. Pictorially, this Green's function is the amplitude for the following process:
\begin{figure}[H]
\centering
\feynmandiagram [vertical=a to b] {
	a -- [photon, edge label=\(p'{, }\tau\)] b [blob], 
	f1 [particle=\(0\)]-- [fermion] b -- [fermion, edge label = \(p\)] i2 [particle=\(t\)], 
};
\end{figure}

In practice, we will often project the initial and final state to some definite polarization in Dirac space, for example examining  
the propagation amplitude for a regular particle (not antiparticle) with spin up. Our Green's function is:
\begin{equation}
	G(t, \tau; p, p'; \mathcal P) := \sum_{\vec x_1, \vec x_2} e^{-i\vec p'\cdot (\vec x_2 - \vec x_1)}e^{-i\vec p\cdot\vec x_1}\mathcal P_{\beta\alpha}
	\langle\Omega | T\{\chi_\alpha(\vec x_2, t)\mathcal O (\vec x_1, \tau)\bar\chi_\beta(0)\}| \Omega\rangle
\end{equation}
where $\mathcal O$ is the operator we are inserting and $\mathcal P_{\beta\alpha}$ is a projection matrix\footnote{In the textbook I've been using, 
they represent this with $\Gamma$, then also represent the operator by $\bar q\Gamma q$. I'll use $\mathcal P$ for the projection to avoid 
confusion.} onto the corresponding polarization in 
spinor space. Typically, the operators that we use will be of the form:
\begin{equation}
	\mathcal O(x) = \bar q(x) \Gamma q(x)
\end{equation}
where $\Gamma$ is some combination of $\gamma$ matrices and covariant derivative. For example, we use $\Gamma = \gamma^\mu$ for the 
electromagnetic current. 

This is the object we will compute explicitly on the lattice through the real-space relation:
\begin{align}
	\langle\Omega | T\{\chi_\alpha(\vec x_2, t)\mathcal O(\vec x_1, \tau)\bar\chi_\beta(0)\} |\Omega\rangle &= \frac{1}{Z}\int DU D\psi D\bar\psi
	\,e^{-S_E[U]}\chi_\alpha(\vec x_2, t)\mathcal O(\vec x_1, \tau)\bar\chi_\beta(0) \\
	&= \frac{1}{Z}\int DU e^{-S_E[U]} \langle\chi_\alpha(\vec x_2, t)\mathcal O(\vec x_1, \tau)\bar\chi_\beta(0)\rangle_F
\end{align}
where $\langle\cdots\rangle_F$ is the fermionic average (obtained from integrating out the fermionic fields). 
To compute this on the lattice, we will need to pick a particular form for the interpolator and work through the fermionic parts of the integral, and 
will then be able to evaluate the gauge field integral by sampling the distribution and working out an averaged sum. This is all put together in the 
following formula for the Green's function:
\begin{equation}
	G(t, \tau; p, p'; \mathcal P) = \sum_U\sum_{\vec x_1, \vec x_2} e^{-i\vec p\cdot (\vec x_2 - \vec x_1)} e^{-i\vec p'\cdot\vec 
	x_1}\mathcal P_{\beta\alpha}\langle\chi_\alpha(\vec x_2, t)\mathcal O(\vec x_1, \tau)\bar\chi_\beta(0)\rangle_F~
	\label{eq:greens_fn_reduced}
\end{equation}
where it is understood that $\sum_U$ will sum on the gauge field configurations sampled according to the probability measure $DU e^{-S_E[U]}$. 

In practice, this is as far as we can get without selecting specific operators. Recall that we may calculate the fermionic average by using Wick's 
theorem once we have a particular choice of the operators for the problem. We will do an example of this for the pion and the proton in a later 
section, once we discuss how to tie up the sources in the correct manner. 

When we compute the three point function, there are two ways to do this: \textbf{through the sink}, and \textbf{through the operator}. This amounts to 
the following: explicitly, we need to compute:

\subsection{Three Point Function Example: Pions and Protons}

The pion is the simpler case, and will be easier to work through first before considering the proton. Its interpolator is:
\begin{equation}
	\chi(x) = \bar d(x)\gamma_5 u(x)
\end{equation}
Note that because meson interpolators are Dirac singlets, we do not include the projection matrix $\mathcal P_{\beta\alpha}$, and we will be 
computing a three point function which has no free Dirac indices. We will work with an current insertion interacting with the proton, in which the 
operator $\mathcal O$ has the following form: 
\begin{equation}
	\mathcal O(x) = \bar u(x)\Gamma u(x)
\end{equation}
We compute the Green's function by starting with the fermionic average in Equation~\ref{eq:greens_fn_reduced} and applying Wick's theorem:
\begin{align}
	\langle\chi(\vec x_2, t) & \mathcal O(\vec x_1, \tau)\bar\chi(0)\rangle_F \nonumber\\ 
	&= \langle \bar d_\alpha^a(x_2)\left(\gamma_5\right)_{\alpha\beta}
	 u_\beta^a(x_2)\bar u_\gamma^b(x_1)\Gamma_{\gamma\sigma}u_\sigma^b(x_1)\bar u_\rho^c(0)\left(\gamma_5\right)_{\rho\delta}
	 d_\delta^c(0)\rangle_F \nonumber \\
	 &= (\gamma_5)_{\alpha\beta}\Gamma_{\gamma\sigma}(\gamma_5)_{\rho\delta}\left[
	U_{\beta\gamma}^{ab}(x_2 | x_1) U_{\sigma\rho}^{bc}(x_1 | 0) D_{\delta\alpha}^{ca}(0 | x_2)
	- U_{\beta\rho}^{ac}(x_2 | 0) U_{\sigma\gamma}^{bb}(x_1 | x_1) D_{\delta\alpha}^{ca}(0 | x_2)
	 \right] \nonumber \\
	 &= tr\{D(0 | x_2)\gamma_5 U(x_2 | x_1) \Gamma U(x_1 | 0) \gamma_5\} - tr\{D(0 | x_2)\gamma_5 U(x_2 | x_1) \gamma_5 \}
	 tr\{U(x_1 | x_1)\Gamma\}
	 \label{eq:pion_threept}
\end{align}
where $U(n|m) = D_u^{-1}(n|m)$ is the up quark propagator and $D(n|m) = D_d^{-1}(n|m)$ is the down quark propagator. Notice that we convert 
this into a trace via:
\begin{equation}
	tr\{ABC\} = (ABC)_{\alpha\alpha} = A_{\alpha\beta} B_{\beta\gamma} C_{\gamma\alpha}
\end{equation}
and in our case above, $tr$ denotes \textbf{both} a spin and color trace. 


\subsection{Correlation Function Ratios}

\subsection{Form Factors}

\subsection{GPDs}

\end{document}