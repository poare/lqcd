\documentclass[11pt, oneside]{article}   	% use "amsart" instead of "article" for AMSLaTeX format
\usepackage[margin = 1in]{geometry}                		% See geometry.pdf to learn the layout options. There are lots.
\geometry{letterpaper}                   		% ... or a4paper or a5paper or ... 
%\geometry{landscape}                		% Activate for rotated page geometry
%\usepackage[parfill]{parskip}    		% Activate to begin paragraphs with an empty line rather than an indent
\usepackage{graphicx}				% Use pdf, png, jpg, or eps§ with pdflatex; use eps in DVI mode
								% TeX will automatically convert eps --> pdf in pdflatex		
\usepackage{amssymb}
\usepackage{amsmath}
\usepackage[shortlabels]{enumitem}
\usepackage{float}
\usepackage{tikz-cd}
\usepackage[compat=1.0.0]{tikz-feynman}   %note you need to compile this in LuaLaTeX for diagrams to render correctly
\usepackage{slashed}

\usepackage{amsthm}
\theoremstyle{definition}
\newtheorem{definition}{Definition}[section]
\newtheorem{theorem}{Theorem}[section]
\newtheorem{corollary}{Corollary}[theorem]
\newtheorem{lemma}[theorem]{Lemma}

\newcommand{\N}{\mathbb{N}}
\newcommand{\R}{\mathbb{R}}
\newcommand{\Z}{\mathbb{Z}}
\newcommand{\Q}{\mathbb{Q}}

%SetFonts

%SetFonts


\title{Hadron Structure}
\author{Patrick Oare}
\date{}							% Activate to display a given date or no date

\begin{document}
\maketitle

\section{Deep Inelastic Scattering (DIS)}

This document will begin by studying the Deep Inelastic Scattering (DIS) experiment, which is the process of a high energy electron scattering 
off of a proton via the exchange of a photon, as illustrated in Figure~\ref{fig:dis}. DIS is an example which will allow us to illustrate most of the 
interesting features of hadron structure, and we will be able to define many of our objects of interest with regards to this type of scattering 
specifically. 

\begin{figure}[H]
\centering
\feynmandiagram [vertical=a to b] {
  i1 [particle=\(e^{-}\)] -- [fermion, edge label'=\(p\)] a -- [fermion, edge label'=\(p'\)] i2 [particle=\(e^{-}\)],
  a -- [photon, edge label=\(q\)] b [blob],
  f1 [particle=\(X\)] -- [anti fermion] b -- [anti fermion, edge label'=\(P\)] f2 [particle=\(p^+\)],
};
\caption{The Deep Inelastic Scattering (DIS) diagram}~
\label{fig:dis}
\end{figure}
%\begin{figure}[H]
%\centering
%    \begin{tikzpicture}
%    \begin{feynman}
%    \vertex (a) {\(e^{-}\)};
%    \vertex [below right=of a] (b);
%    \vertex [above right=of b] (f1) {\(e^-\)};
%    \vertex [below=of b] (c);
%    \vertex [below left=of c] (f2) {\(p^+\)};
%    \vertex [below right=of c] (f3) {\(X\)};
%    \diagram* {
%    (a) -- [fermion] (b) -- [fermion] (f1),
%    (b) -- [boson, edge label'=\(q\)] (c),
%    (c) -- [anti fermion] (f2),
%    (c) -- [fermion] (f3),
%    };
%    \end{feynman}
%    \end{tikzpicture}
%\end{figure}

As the energy of the initial electron is increased, the transfer momentum $q$ becomes larger and larger until it is able 
to break up the proton into any particle that it can be broken into. As a result, the scattering starts off as elastic until it 
reaches some energy threshold, then becomes highly inelastic as the proton is smashed into different particles. Without 
any knowledge of the interaction that occurs at the bottom vertex, we can write down a surprising amount of information 
about the scattering process. The amplitude is:
\begin{equation}
	i\mathcal M = -ie\overline u(p')\gamma^\mu u(p) \frac{-ig_{\mu\nu}}{q^2}\hat{\mathcal M^\nu}(q) = 
	-\frac{e}{q^2}\overline u(p')\gamma^\mu u(p)\hat{\mathcal M}_\mu(q)
\end{equation}
where $i\hat{\mathcal M}(q)^\mu$ is the amplitude for the photon interacting with the proton and breaking it into a final 
state $|X\rangle$:
\begin{equation}
i\hat{\mathcal M}^\mu(q) =
\begin{gathered}
\feynmandiagram [small, vertical=a to b] {
  a [particle=\(\mu\)] -- [photon, edge label=\(q\)] b [blob],
  f1 [particle=\(X\)] -- [anti fermion] b -- [anti fermion] f2 [particle=\(p^+\)],
};
\end{gathered}~
\label{eq:dis_hadronic_diagram}
\end{equation}
If we want to calculate the unpolarized cross section, then we need the spin averaged matrix element:
\[
	\overline{|\mathcal M|^2} = \frac{1}{2}\sum_{spins, X}|\mathcal M|^2 = \frac{e^2}{2q^4}\left(\sum_{sr}\overline u_r(k')\gamma^\mu 
	u_s(k)\overline u_s(k)\gamma^\nu u_s(k')\right)\left(\sum_{spins}\mathcal M_\mu(q)\mathcal M_\nu^*(q)\right)
\]
\begin{equation}
	= \frac{e^2}{2q^4} tr\left[\slashed{k'}\gamma^\mu \slashed{k}\gamma^\nu\right] \left(\sum_{spins}\mathcal M_\mu(q)\mathcal M_\nu^*(q)\right)~
	\label{eq:dis_spin_averaged}
\end{equation}
We can integrate this over phase space to get a differential cross section in the lab frame, which we use to define two tensors:
\begin{equation}
	\left(\frac{d\sigma}{d\Omega dE'}\right)_{lab} = \frac{\alpha_e^2}{4\pi m_p q^4}L^{\mu\nu} W_{\mu\nu}
\end{equation}

The first tensor $L{\mu\nu}$ is the \textbf{leptonic tensor} and describes all aspects of the scattering related to the scattering 
of the initial and final electrons, and how they interact with the photon. Despite not knowing much about the scattering process 
$\gamma^* p^+\rightarrow X$, we can still explicitly write this tensor down by working through the math starting from 
Equation~\ref{eq:dis_spin_averaged}:
\begin{equation}
	L^{\mu\nu} := \frac{1}{2}tr\left[\slashed{k'}\gamma^\mu\slashed{k}\gamma^\nu\right] = 2(k'^\mu k^\nu + k'^\nu k^\mu - k\cdot k' g^{\mu\nu})
\end{equation}

The second tensor $W_{\mu\nu}$ is the \textbf{hadronic tensor}, and describes the physics in the hadronic part of the process, 
namely when the proton interacts with the mediating photon. Although we do not know many details about the process, we can still 
use general principles of symmetry to make some headway into describing the physics, without explicitly knowing information 
about the final state or the interaction. Explicitly, the hadronic tensor is:
\begin{equation}
	e^2\epsilon_\mu\epsilon_\nu^* W^{\mu\nu} := \frac{1}{2}\sum_{X, spins} \int d\Pi_X (2\pi)^2\delta^4\left(\sum p\right)
	|\mathcal M(\gamma^*p^+\rightarrow X)|^2~
	\label{eq:hadronic_tensor}
\end{equation}
and depends explicitly on the amplitude for the virtual proton to scatter of the proton and produce the state $|X\rangle$, which is 
yet unknown. Now, we will use this equation and this decomposition to explore the physics of this problem, and will show that 
we can predict a surprising amount without knowing the details of the interaction in Equation~\ref{eq:dis_hadronic_diagram}.

\subsection{Form Factors}

We can exploit the symmetry of the problem to heavily constrain the form of the hadronic tensor, and 
extract physics from this constrained form. This is the \textbf{method of form factors}. We know the following things about 
$W^{\mu\nu}$:
\begin{enumerate}
	\item It may only depend on the momentum $q^\mu$ and $P^\mu$, since the external momenta in the state $|X\rangle$ are 
	integrated over in the phase space integral. 
	\item It must be symmetric, i.e. $W^{\mu\nu} = W^{\nu\mu}$, because we assume the initial photon is unpolarized. 
	\item It must obey the Ward identity applied to the diagram in Equation~\ref{eq:dis_hadronic_diagram}, so $q_\mu W^{\mu\nu} = 0$. 
\end{enumerate}

Items 1 and 2 on the list say that we can obtain a general form for $W^{\mu\nu}$ by forming all symmetric rank 2 combinations 
of the Lorentz vectors $q^\mu$ and $P^\mu$, as well as the metric $g^{\mu\nu}$, and examining all linear combinations of 
them. This gives 3 parameters we can vary in the linear combination for the general form for $W^{\mu\nu}$. 
However, the Ward identity $q_\mu W^{\mu\nu} = 0$ constrains the form these combinations can take on, and so we actually 
only have 2 coefficients we can vary in the linear combination, which we will call $W_1$ and $W_2$. These coefficients are called 
\textbf{form factors}, and we can explicitly write the most general form of $W^{\mu\nu}$ consistent with our constraints:
\begin{equation}
	W^{\mu\nu} = \left(-g^{\mu\nu} + \frac{q^\mu q^\nu}{q^2}\right) W_1 + \left(P^\mu - \frac{P\cdot q}{q^2}q^\mu\right)\left(P^\nu - 
	\frac{P\cdot q}{q^2}q^\nu\right) W_2~
	\label{eq:form_factors}
\end{equation}

The form factors $W_1$ and $W_2$ are Lorentz scalars, and therefore must be functions of the Lorentz scalars we can generate with 
our dynamical variables $q^\mu$ and $P^\mu$. The three combinations that we can create are $q^2$, $P^2$, and $P\cdot q$. Note that 
$P^2 = m_p^2$ is not a dynamical variable because we assume the initial photon is on shell, but since we do not make this assumption 
with the photon, $q^2$ is a dynamical variable that our form factors can depend on. Thus we can write the form factors as functions 
of $q^2$ and $P\cdot q$. 

To massage this into a nicer form, let $Q := \sqrt{-q^2}$ be the energy scale of the collision, and define the dimensionless ratio:
\begin{equation}
	x := \frac{Q^2}{2 P\cdot q}
\end{equation}
This is called the \textbf{Bjorken variable}, and is an important variable in describing hadron structure. Physically, you can think of $x$ as a 
momentum fraction. So, we will consider our form factors as functions of $x$ and $Q$:
\begin{equation}
	W_1 = W_1(x, Q)\;\;\;\;\;\;\;\;\;\;\;\;\;\;\;\;\;\;\;\; W_2 = W_2(x, Q)
\end{equation}

Using the general form Equation~\ref{eq:form_factors} of the hadronic tensor, we can plug plug this into our cross section to express 
it explicitly in terms of $W_1$ and $W_2$:
\begin{equation}
	\left(\frac{d\sigma}{d\Omega dE'}\right)_{lab} = \frac{\alpha_e^2}{8\pi E^2\sin^4(\theta / 2)}\left(\frac{m_p}{2} W_2(x, Q)\cos^2\frac{\theta}{2} + 
	\frac{1}{m_p} W_1(x, Q)\sin^2\frac{\theta}{2}\right)~
	\label{eq:sigma_ff}
\end{equation}
This should be pretty remarkable: without knowing anything about the final state $|X\rangle$ or really any details of the hadronic process 
described by the diagram of Equation~\ref{eq:dis_hadronic_diagram}, we have expanded an observable in terms of the two form 
factors $W_1$ and $W_2$. This means that the form factors can be experimentally measured, and then we can use those measurements 
to make other predictions about the process. 

\subsection{The Parton Model}

A \textbf{parton} is a point-like particle which has no composite substructure-- examples of partons are the electron, neutrino, and quarks. 
When nuclear structure was being studied, the proton was originally believed to be a parton as well until experiments like DIS showed 
that it instead has a substructure of quarks and gluons. 

Feynman coined the \textbf{parton model} of the proton, in which he assumed the proton was made up of constituent 
partons that interact weakly and are almost free particles. Suppose that we assume the proton is made up of partons of mass 
$\{m_i\}_{i}$. In DIS, we can assume the photon interacts with a single parton, say parton $j$. Then we can describe this process 
with the diagram in Figure~\ref{fig:dis} where we replace the proton line with a parton line with incoming momentum $p_j$ and external 
momentum $p_j'$. Momentum conservation gives us $p_j + q = p_j'$, and squaring both sides we get:
\begin{equation}
	\frac{Q^2}{2p_j\cdot q} = 1
\end{equation}
Since the parton is a constituent of the proton with momentum $P$, assume that it has a fraction $\xi$ of the proton's 
momentum $P$, i.e. $p_j = \xi P$. Then using the equation above, we find the Bjorken variable is:
\begin{equation}
	x = \frac{Q^2}{2P\cdot q} = \xi \frac{Q^2}{2p_j\cdot q} = \xi
\end{equation}
so with these assumptions, the Bjorken $x$ is exactly the momentum fraction of the parton which is involved in the scattering 
process. 

Note that in the physical world described fully by QCD, the Bjorken variable is not the momentum fraction of the 
parton. This will be approximately true, \textbf{but in general $x$ and $\xi$ are different, although they are related}. Thinking of the physical 
interpretation of $x$ as the momentum fraction is helpful, but not exactly accurate. 

In the actual proton, partons are not free but interact. Assuming they interact via the electromagnetic force, we can 
examine the process $e^-q\rightarrow e^-q$ through a photon exchange, where $q$ is the parton. When we studied IR divergences, 
we calculated this for the $e^-e^+\rightarrow\mu^-\mu^+$ scattering, and found that the form factor $F_1(q^2)$ ran as $F_1(q^2)\propto 
log(Q^2)$ and $log^2(Q^2)$ when the initial momentum was fixed (i.e. when we vary $x$). This weak running of the form factors 
as we vary $Q$ at fixed $x$ is called \textbf{Bjorken scaling} and applies in the parton model as well: when we work at fixed $x$: 
the form factors stay relatively constant as we vary the energy scale of the collision. 

Another key ingredient of the parton model is an object known as a \textbf{Parton Distribution Function (PDF)}. PDFs are functions 
$f_i(\xi)d\xi$ which give the probability density for scattering off of parton $i$ having momentum fraction $\xi$. Since the Bjorken 
variable has an interpretation as the momentum fraction of the proton which is contained by the parton, we will write out PDFs as 
$f_i(x)dx$. The parton model assumes that we can factorize the DIS cross section into partonic ones:
\begin{equation}
	\sigma(e^- P^+\rightarrow e^- X) = \sum_i\int_0^1dx f_i(x)\hat{\sigma}(e^- p_i\rightarrow e^- X)~
	\label{eq:sigma_pdf}
\end{equation}
Here the sum runs over all partons $p_i$ contained in the proton, and $\hat\sigma$ is the cross section for the partonic process. 

Now, we assume that the partons only interact with the proton through QED and have charges $Q_i$ ($Q_i$ is the electric charge of 
different types of quarks). This allows us to use the Rosenbluth formula for spin 1/2 QED scattering with form factors $F_1\equiv 1$ and 
$F_2\equiv 0$. Plugging these into Equation~\ref{eq:sigma_pdf} and turning it into a differential cross section, we can compare it to 
Equation~\ref{eq:sigma_ff} to read off the form factors $W_1$ and $W_2$:
\begin{equation}
	W_1(x, Q) = 2\pi\sum_i Q_i^2 f_i(x) \hspace{3cm}W_2(x, Q) = \frac{8\pi x^2}{Q^2}\sum_i Q_i^2 f_i(x)~
	\label{eq:ff_decomp}
\end{equation}
The relation between $W_1$ and $W_2$ is called the \textbf{Callan-Gross relation}, and was used as experimental evidence for the 
parton model. It is a direct consequence of the fact that quarks have spin 1/2. The relation is:
\begin{equation}
	W_1(x, Q) = \frac{Q^2}{4x^2} W_2(x, Q)
\end{equation}

Equation~\ref{eq:ff_decomp} allow us to write out the proton form factors in terms of PDFs (note here we adopt the notation that the pdf $q(x) := 
f_q(x)$, so for example we use $u(x)$ to denote $f_u(x)$). We can see that the proton has $W_1$ form factor:
\begin{equation}
	W_1(x, Q) = 2\pi\left(\frac{4}{9} u + \frac{4}{9}\bar u + \frac{1}{9} d + \frac{1}{9}\bar d+ \frac{4}{9} s + \frac{4}{9}\bar s + ...\right)
\end{equation}
This can then be used to determine the form factors of the neutron, using isospin symmetry $u\leftrightarrow d$. 

PDFs must satisfy certain constraints to be valid probability distributions: namely, they must satisfy certain \textbf{sum rules} related to 
particle number conservation. As an example of this, consider the up quarks in the proton. While at any given time there are 2 valence quarks, 
we can also have $q\overline q$ production to create sea quarks. Since the total number of up quarks is conserved ($N_u + N_{\bar u} = 2$) 
we must have:
\begin{equation}
	\int_0^1dx \left(f_{u}(x) - f_{\bar u}(x)\right) = 2
\end{equation}
because for a single quark PDF, the total integral $\int_0^1dx q(x) = N_q$. Similar rules hold in the proton for down ($N_d = 1$) and other 
flavor ($N_f = 0$ for $f = s, c, b, t$) quarks. In general, a sum rule will apply anytime there is a conserved particle number. The proton is a bound 
state with specific quantum numbers for its number of (each flavor of) quarks, which implies the sum rules for the quark PDFs. Since specifying a 
bound hadron or meson state also specifies its quantum numbers, this means that \textit{each bound QCD state will come with its own set of sum 
rules} to reflect its general make-up. Note because gluon number is not conserved, there is \textit{no general sum rule for the gluon PDFs in a QCD 
bound state}. 

Finally, there is an additional sum rule for momentum conservation:
\begin{equation}
	\sum_i \int_0^1 dx\, x f_i(x) = 1
\end{equation}

\newpage
\section{Renormalization of PDFs: The DGLAP Equations}

Bjorken scaling holds quite well in actual experiments, but because the parton model is not a completely accurate physical theory (it does not 
allow for interactions between the constituent partons which are found in actual QCD), Bjorken scaling is violated to some extent. Examining the 
amount to which Bjorken scaling is violated will lead us to the DGLAP equations, which describe how parton PDFs mix under renormalization. 
To study this, define a modified hadronic tensor $\hat{W}_{\mu\nu}$, which is the definition in Equation~\ref{eq:hadronic_tensor} with $p^+$ 
replaced with a parton $p_i$. Similarly, let $z_i := Q^2 / 2p_i\cdot q$ be the counterpart of the Bjorken variable for a specific parton. Because 
$p_i = \xi P$ with $P$ the total momentum of the incoming proton, we see that $x = \xi z_i$. 

We can use this information to relate $W_{\mu\nu}$ to $\hat W_{\mu\nu}$. We can decompose the diagram defining $W_{\mu\nu}$ into 
a sum of diagrams where the photon interacts with each parton, i.e. we sum over all partons. We must consider the partons carrying an 
arbitrary momentum fraction of the proton constrained by $x = \xi z_i$, so we integrate over $d\xi f_i(\xi)$ and $dz_i$ with an additional 
$\delta$ function to enforce that $x = \xi z_i$. Note whenever we integrate over $\xi$, we must integrate $d\xi f_i(\xi)$ because $f_i(\xi)$ 
is the density that describes how the momentum of the parton is distributed. This amounts to:
\begin{equation}
	W_{\mu\nu}(x, Q) = \sum_i\int_0^1 dz\int_0^1 d\xi f_i(\xi) \hat{W}_{\mu\nu}(z, Q)\delta(x - \xi z) = \sum_i\int_x^1 \frac{d\xi}{\xi} f_i(\xi)
	\hat{W}_{\mu\nu}\left(\frac{x}{\xi}, Q\right)
\end{equation}

We next take the trace of the hadronic tensor, and use that to define the form factor $W_0$:
\begin{equation}
	W_0(x, Q) := -g_{\mu\nu}W^{\mu\nu} = 3W_1(x, Q) - \left(m_p^2 + \frac{Q^2}{4x^2}\right)W_2(x, Q)\xrightarrow{Q >> m_p} 3W_1 
	- \frac{Q^2}{4x^2}W_2
\end{equation}
Plugging in the Callan-Gross relation, we see that for large $Q / x$ we have $W_0 = 2W_1$ to first order, and using Equation~\ref{eq:ff_decomp} 
allows us to relate $W_0$ directly to the PDFs:
\begin{equation}
	W_0(x, Q) = 4\pi\sum_i Q_i^2 f_i(x)
\end{equation}
Since $W_0$ is easily calculated in perturbation theory (it is essentially just the sum of unpolarized cross sections $\gamma^* P\rightarrow X$), we 
will use this \textit{to define the PDF $f_i(x)$}. In particular, we can compute $W_0$ to higher orders in perturbation theory to determine 
the PDFs $f_i(z)$. Explicitly, we will compute the next to leading order contributions to the cross section $\gamma^* \rho\rightarrow \rho$, where 
$\rho$ is a parton, and define this to be $\hat W_0^{NLO}$. 

This calculation will be very similar to the computation of IR divergences in the $e^+e^-\rightarrow\mu^+\mu^-$ cross section. Recall in that 
case we had to compute real emission graphs in addition to the regular cross section to cancel the IR divergences, and when we summed 
the loop diagrams with the emission graphs to each order in perturbation theory, the divergences vanished. We must do this in the same 
way for this computation. 

The next to leading order correction to $W_0$ can be calculated by considering the following amplitudes. First, the tree level vertex:
\begin{equation}
    i\mathcal M_V =
    \begin{gathered}
        \feynmandiagram [small, vertical=a to b] {
                a [particle=\(\gamma^*\)] -- [photon, edge label=\(q\)] b,
                	f1 [particle=\(\rho\)] -- [anti fermion, edge label' = \(p_f\)] b,
		b -- [anti fermion, edge label' = \(p_i\)] f2 [particle=\(\rho\)],
        };
\end{gathered}~
\label{eq:tree_vertex}
\end{equation}
Next, the vertex correction from an incoming parton:
\begin{equation}
    i\mathcal M_V =
    \begin{gathered}
        \feynmandiagram [vertical=a to b] {
                a [particle=\(\gamma^*\)] -- [photon, edge label=\(q\)] b,
                c -- [anti fermion] b -- [anti fermion] d -- [gluon, half right] c,
                	f1 [particle=\(\rho\)] -- [anti fermion, edge label' = \(p_f\)] c,
		d -- [anti fermion, edge label' = \(p_i\)] f2 [particle=\(\rho\)],
        };
\end{gathered}~
\label{eq:virtual_emission}
\end{equation}
The next to leading order contribution in $W_0$ will go as $\mathcal M_0\mathcal M_V^* + h.c.$, i.e. will be the interference terms between 
this diagram and the original vertex. We will call this contribution $\hat W_0^V$. The second contribution to $\hat W_0^{NLO}$ is the real emission 
graphs from tree level parton scattering:
\begin{equation}
    i\mathcal M_R =
    \begin{gathered}
        \feynmandiagram [horizontal=a to b] {
		i2 -- [fermion, edge label' = \(p_i\)] a,
        		i1 [particle=\(\gamma^*\)] -- [photon, edge label' = \(q\)] a,
		a -- [fermion] b,
		b -- [fermion, edge label' = \(p_f\)] f1 [particle = \(\rho\)],
		b -- [gluon, edge label' = \(p_g\)] f2,
        };
\end{gathered}
+
\begin{gathered}
        \feynmandiagram [vertical=a to b] {
        		i1 [particle=\(\gamma^*\)] -- [photon, edge label' = \(q\)] a,
		b -- [gluon, edge label' = \(p_g\)] f2,
		b -- [anti fermion, edge label' = \(p_i\)] f1 [particle = \(\rho\)],
		i2 -- [anti fermion, edge label' = \(p_f\)] a,
		a -- [anti fermion] b,
        };
\end{gathered}
~
\label{eq:real_emission}
\end{equation}
The cross section from considering $\mathcal M_R\mathcal M_R^*$ in this process is the same order as $\hat W_0^V$, and so will contribute 
to $\hat W_0^{NLO}$ as well. We will call this contribution $\hat W_0^R$. So, the total next to leading order correction to the form factor $W_0$ is:
\begin{equation}
	\hat W_0 = \hat W_0^{LO} + \hat W_0^V + \hat W_0^R
\end{equation}

\newpage
\section{General Theory}

\subsection{Form Factors}

As we saw in the previous section, form factors are objects which allow us to probe the structure of composite particles. Formally, 
they are simply expansion coefficients in a given basis of Lorentz vectors, which can be given physical interpretation by 
examining their structure in various limits. 

For another example of this, consider the general QED vertex function $\Gamma^\mu$, which includes radiative corrections. 

\subsection{Parton Distribution Functions (PDFs)}

\subsection{Generalized Parton Distributions (GPDs)}

\newpage
\section{Computational Methods}

\subsection{Three Point Functions}

\subsection{Correlation Function Ratios}

\subsection{Form Factors}

\subsection{GPDs}

\end{document}