\documentclass[11pt, oneside]{article}   	% use "amsart" instead of "article" for AMSLaTeX format
\usepackage[margin = 1in]{geometry}                		% See geometry.pdf to learn the layout options. There are lots.
\geometry{letterpaper}                   		% ... or a4paper or a5paper or ... 
%\geometry{landscape}                		% Activate for rotated page geometry
%\usepackage[parfill]{parskip}    		% Activate to begin paragraphs with an empty line rather than an indent
\usepackage{graphicx}				% Use pdf, png, jpg, or eps§ with pdflatex; use eps in DVI mode
								% TeX will automatically convert eps --> pdf in pdflatex		
\usepackage{amssymb}
\usepackage{amsmath}
\usepackage[shortlabels]{enumitem}
\usepackage{float}
\usepackage{tikz-cd}
\usepackage[compat=1.0.0]{tikz-feynman}   %note you need to compile this in LuaLaTeX for diagrams to render correctly
\usepackage{slashed}

\usepackage{amsthm}
\theoremstyle{definition}
\newtheorem{definition}{Definition}[section]
\newtheorem{theorem}{Theorem}[section]
\newtheorem{corollary}{Corollary}[theorem]
\newtheorem{lemma}[theorem]{Lemma}

\newcommand{\N}{\mathbb{N}}
\newcommand{\R}{\mathbb{R}}
\newcommand{\Z}{\mathbb{Z}}
\newcommand{\Q}{\mathbb{Q}}

%SetFonts

%SetFonts


\title{Hadron Structure}
\author{Patrick Oare}
\date{}							% Activate to display a given date or no date

\begin{document}
\maketitle

\section{Deep Inelastic Scattering (DIS)}

The canonical example of an experimental probe of the parton structure of the proton is the Deep Inelastic Scattering (DIS) 
experiment. This is the process of a high energy electron scattering off of a proton via the exchange of a photon, as 
illustrated in the following diagram. 
\begin{figure}[H]
\centering
\feynmandiagram [vertical=a to b] {
  i1 [particle=\(e^{-}\)] -- [fermion, edge label'=\(p\)] a -- [fermion, edge label'=\(p'\)] i2 [particle=\(e^{-}\)],
  a -- [photon, edge label=\(q\)] b [blob],
  f1 [particle=\(X\)] -- [anti fermion] b -- [anti fermion, edge label'=\(P\)] f2 [particle=\(p^+\)],
};
\caption{The Deep Inelastic Scattering (DIS) diagram}~
\label{fig:dis}
\end{figure}
%\begin{figure}[H]
%\centering
%    \begin{tikzpicture}
%    \begin{feynman}
%    \vertex (a) {\(e^{-}\)};
%    \vertex [below right=of a] (b);
%    \vertex [above right=of b] (f1) {\(e^-\)};
%    \vertex [below=of b] (c);
%    \vertex [below left=of c] (f2) {\(p^+\)};
%    \vertex [below right=of c] (f3) {\(X\)};
%    \diagram* {
%    (a) -- [fermion] (b) -- [fermion] (f1),
%    (b) -- [boson, edge label'=\(q\)] (c),
%    (c) -- [anti fermion] (f2),
%    (c) -- [fermion] (f3),
%    };
%    \end{feynman}
%    \end{tikzpicture}
%\end{figure}

As the energy of the initial electron is increased, the transfer momentum $q$ becomes larger and larger until it is able 
to break up the proton into any particle that it can be broken into. As a result, the scattering starts off as elastic until it 
reaches some energy threshold, then becomes highly inelastic as the proton is smashed into different particles. Without 
any knowledge of the interaction that occurs at the bottom vertex, we can write down a surprising amount of information 
about the scattering process. The amplitude is:
\begin{equation}
	i\mathcal M = -ie\overline u(p')\gamma^\mu u(p) \frac{-ig_{\mu\nu}}{q^2}\hat{\mathcal M^\nu}(q) = 
	-\frac{e}{q^2}\overline u(p')\gamma^\mu u(p)\hat{\mathcal M}_\mu(q)
\end{equation}
where $i\hat{\mathcal M}(q)^\mu$ is the amplitude for the photon interacting with the proton and breaking it into a final 
state $|X\rangle$:
\begin{equation}
i\hat{\mathcal M}^\mu(q) =
\begin{gathered}
\feynmandiagram [small, vertical=a to b] {
  a [particle=\(\mu\)] -- [photon, edge label=\(q\)] b [blob],
  f1 [particle=\(X\)] -- [anti fermion] b -- [anti fermion] f2 [particle=\(p^+\)],
};
\end{gathered}~
\label{eq:dis_hadronic_diagram}
\end{equation}
If we want to calculate the unpolarized cross section, then we need the spin averaged matrix element:
\[
	\overline{|\mathcal M|^2} = \frac{1}{2}\sum_{spins, X}|\mathcal M|^2 = \frac{e^2}{2q^4}\left(\sum_{sr}\overline u_r(k')\gamma^\mu 
	u_s(k)\overline u_s(k)\gamma^\nu u_s(k')\right)\left(\sum_{spins}\mathcal M_\mu(q)\mathcal M_\nu^*(q)\right)
\]
\begin{equation}
	= \frac{e^2}{2q^4} tr\left[\slashed{k'}\gamma^\mu \slashed{k}\gamma^\nu\right] \left(\sum_{spins}\mathcal M_\mu(q)\mathcal M_\nu^*(q)\right)~
	\label{eq:dis_spin_averaged}
\end{equation}
We can integrate this over phase space to get a differential cross section in the lab frame, which we use to define two tensors:
\begin{equation}
	\left(\frac{d\sigma}{d\Omega dE'}\right)_{lab} = \frac{\alpha_e^2}{4\pi m_p q^4}L^{\mu\nu} W_{\mu\nu}
\end{equation}

The first tensor $L{\mu\nu}$ is the \textbf{leptonic tensor} and describes all aspects of the scattering related to the scattering 
of the initial and final electrons, and how they interact with the photon. Despite not knowing much about the scattering process 
$\gamma^* p^+\rightarrow X$, we can still explicitly write this tensor down by working through the math starting from 
Equation~\ref{eq:dis_spin_averaged}:
\begin{equation}
	L^{\mu\nu} := \frac{1}{2}tr\left[\slashed{k'}\gamma^\mu\slashed{k}\gamma^\nu\right] = 2(k'^\mu k^\nu + k'^\nu k^\mu - k\cdot k' g^{\mu\nu})
\end{equation}

The second tensor $W_{\mu\nu}$ is the \textbf{hadronic tensor}, and describes the physics in the hadronic part of the process, 
namely when the proton interacts with the mediating photon. Although we do not know many details about the process, we can still 
use general principles of symmetry to make some headway into describing the physics, without explicitly knowing information 
about the final state or the interaction. Explicitly, the hadronic tensor is:
\begin{equation}
	e^2\epsilon_\mu\epsilon_\nu^* W^{\mu\nu} := \frac{1}{2}\sum_{X, spins} \int d\Pi_X (2\pi)^2\delta^4\left(\sum p\right)
	|\mathcal M(\gamma^*p^+\rightarrow X)|^2
\end{equation}
and depends explicitly on the amplitude for the virtual proton to scatter of the proton and produce the state $|X\rangle$, which is 
yet unknown. Now, we will use this equation and this decomposition to explore the physics of this problem, and will show that 
we can predict a surprising amount without knowing the details of the interaction in Equation~\ref{eq:dis_hadronic_diagram}.

\subsection{Form Factors}

We can exploit the symmetry of the problem to heavily constrain the form of the hadronic tensor, and 
extract physics from this constrained form. This is the \textbf{method of form factors}. We know the following things about 
$W^{\mu\nu}$:
\begin{enumerate}
	\item It may only depend on the momentum $q^\mu$ and $P^\mu$, since the external momenta in the state $|X\rangle$ are 
	integrated over in the phase space integral. 
	\item It must be symmetric, i.e. $W^{\mu\nu} = W^{\nu\mu}$, because we assume the initial photon is unpolarized. 
	\item It must obey the Ward identity applied to the diagram in Equation~\ref{eq:dis_hadronic_diagram}, so $q_\mu W^{\mu\nu} = 0$. 
\end{enumerate}

Items 1 and 2 on the list say that we can obtain a general form for $W^{\mu\nu}$ by forming all symmetric rank 2 combinations 
of the Lorentz vectors $q^\mu$ and $P^\mu$, as well as the metric $g^{\mu\nu}$, and examining all linear combinations of 
them. This gives 3 parameters we can vary in the linear combination for the general form for $W^{\mu\nu}$. 
However, the Ward identity $q_\mu W^{\mu\nu} = 0$ constrains the form these combinations can take on, and so we actually 
only have 2 coefficients we can vary in the linear combination, which we will call $W_1$ and $W_2$. These coefficients are called 
\textbf{form factors}, and we can explicitly write the most general form of $W^{\mu\nu}$ consistent with our constraints:
\begin{equation}
	W^{\mu\nu} = \left(-g^{\mu\nu} + \frac{q^\mu q^\nu}{q^2}\right) W_1 + \left(P^\mu - \frac{P\cdot q}{q^2}q^\mu\right)\left(P^\nu - 
	\frac{P\cdot q}{q^2}q^\nu\right) W_2~
	\label{eq:form_factors}
\end{equation}

The form factors $W_1$ and $W_2$ are Lorentz scalars, and therefore must be functions of the Lorentz scalars we can generate with 
our dynamical variables $q^\mu$ and $P^\mu$. The three combinations that we can create are $q^2$, $P^2$, and $P\cdot q$. Note that 
$P^2 = m_p^2$ is not a dynamical variable because we assume the initial photon is on shell, but since we do not make this assumption 
with the photon, $q^2$ is a dynamical variable that our form factors can depend on. Thus we can write the form factors as functions 
of $q^2$ and $P\cdot q$. 

To massage this into a nicer form, let $Q := \sqrt{-q^2}$ be the energy scale of the collision, and define the dimensionless ratio:
\begin{equation}
	x := \frac{Q^2}{2 P\cdot q}
\end{equation}
This is called the \textbf{Bjorken variable}, and is an important variable in describing hadron structure. Physically, you can think of $x$ as a 
momentum fraction. So, we will consider our form factors as functions of $x$ and $Q$:
\begin{equation}
	W_1 = W_1(x, Q)\;\;\;\;\;\;\;\;\;\;\;\;\;\;\;\;\;\;\;\; W_2 = W_2(x, Q)
\end{equation}

Using the general form Equation~\ref{eq:form_factors} of the hadronic tensor, we can plug plug this into our cross section to express 
it explicitly in terms of $W_1$ and $W_2$:
\begin{equation}
	\left(\frac{d\sigma}{d\Omega dE'}\right)_{lab} = \frac{\alpha_e^2}{8\pi E^2\sin^4(\theta / 2)}\left(\frac{m_p}{2} W_2(x, Q)\cos^2\frac{\theta}{2} + 
	\frac{1}{m_p} W_1(x, Q)\sin^2\frac{\theta}{2}\right)
\end{equation}
This should be pretty remarkable: without knowing anything about the final state $|X\rangle$ or really any details of the hadronic process 
described by the diagram of Equation~\ref{eq:dis_hadronic_diagram}, we have expanded an observable in terms of the two form 
factors $W_1$ and $W_2$. This means that the form factors can be experimentally measured, and then we can use those measurements 
to make other predictions about the process. 

\subsection{The Parton Model}

A \textbf{parton} is a point-like particle which has no composite substructure-- examples of partons are the electron, neutrino, and quarks. 
When nuclear structure was being studied, the proton was originally believed to be a parton as well until experiments like DIS showed 
that it instead has a substructure of quarks and gluons. 

Feynman coined the \textbf{parton model} of the proton, in which he assumed the proton was made up of constituent 
partons that interact weakly and are almost free particles. Suppose that we assume the proton is made up of partons of mass 
$\{m_i\}_{i}$. In DIS, we can assume the photon interacts with a single parton, say parton $j$. Then we can describe this process 
with the diagram in Figure~\ref{fig:dis} where we replace the proton line with a parton line with incoming momentum $p_j$ and external 
momentum $p_j'$. Momentum conservation gives us $p_j + q = p_j'$, and squaring both sides we get:
\begin{equation}
	\frac{Q^2}{2p_j\cdot q} = 1
\end{equation}
Since the parton is a constituent of the proton with momentum $P$, assume that it has a fraction $\xi$ of the proton's 
momentum $P$, i.e. $p_j = \xi P$. Then using the equation above, we find the Bjorken variable is:
\begin{equation}
	x = \frac{Q^2}{2P\cdot q} = \xi \frac{Q^2}{2p_j\cdot q} = \xi
\end{equation}
so with these assumptions, the Bjorken $x$ is exactly the momentum fraction of the parton which is involved in the scattering 
process. 

In the actual proton, partons are not free but interact. Assuming they interact via the electromagnetic force, we can 
examine the process $e^-q\rightarrow e^-q$ through a photon exchange, where $q$ is the parton. When we studied IR divergences, 
we calculated this for the $e^-e^+\rightarrow\mu^-\mu^+$ scattering, and found that the form factor $F_1(q^2)$ ran as $F_1(q^2)\propto 
log(Q^2)$ and $log^2(Q^2)$ when the initial momentum was fixed (i.e. when we vary $x$). This weak running of the form factors 
as we vary $Q$ at fixed $x$ is called \textbf{Bjorken scaling} and applies in the parton model as well: when we work at fixed $x$, 
the form factors stay relatively constant as we vary the energy scale of the collision. 

\newpage
\section{General Theory}

\subsection{Form Factors}

As we saw in the previous section, form factors are objects which allow us to probe the structure of composite particles. Formally, 
they are simply expansion coefficients in a given basis of Lorentz vectors, which can be given physical interpretation by 
examining their structure in various limits. 

For another example of this, consider the general QED vertex function $\Gamma^\mu$, which includes radiative corrections. 

\subsection{Parton Distribution Functions (PDFs)}

\subsection{Generalized Parton Distributions (GPDs)}

\section{Computational Methods}

\subsection{Three Point Functions}

\subsection{Correlation Function Ratios}

\subsection{Form Factors}

\subsection{GPDs}

\end{document}