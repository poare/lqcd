\documentclass[11pt, oneside]{article}   	% use "amsart" instead of "article" for AMSLaTeX format
\usepackage[margin = 1in]{geometry}                		% See geometry.pdf to learn the layout options. There are lots.
\geometry{letterpaper}                   		% ... or a4paper or a5paper or ... 
%\geometry{landscape}                		% Activate for rotated page geometry
%\usepackage[parfill]{parskip}    		% Activate to begin paragraphs with an empty line rather than an indent
\usepackage{graphicx}				% Use pdf, png, jpg, or eps§ with pdflatex; use eps in DVI mode
								% TeX will automatically convert eps --> pdf in pdflatex		
\usepackage{amssymb}
\usepackage{amsmath}
\usepackage[shortlabels]{enumitem}
\usepackage{float}
\usepackage{tikz-cd}

\usepackage{amsthm}
\theoremstyle{definition}
\newtheorem{definition}{Definition}[section]
\newtheorem{theorem}{Theorem}[section]
\newtheorem{corollary}{Corollary}[theorem]
\newtheorem{lemma}[theorem]{Lemma}

\newcommand{\N}{\mathbb{N}}
\newcommand{\R}{\mathbb{R}}
\newcommand{\Z}{\mathbb{Z}}
\newcommand{\Q}{\mathbb{Q}}

%SetFonts

%SetFonts


\title{Hypercubic Symmetry}
\author{Patrick Oare}
\date{}							% Activate to display a given date or no date

\begin{document}
\maketitle

These notes will briefly discuss the representation theory of the hypercubic group $H(4)$. We will begin by discussing 
the cubic group $H(3)$, as it is easier to visualize and many of its properties are shared by its larger cousin $H(4)$. 
We will then characterize $H(4)$ as a group, and classify its irreps via studying its characters. Finally, we will focus on 
some of the more important representations and show how a rank 2 tensor decomposes into the irreps of $H(4)$. 

\section{The Cubic Group}

An easier case to begin studying this subject is by studying the cubic group, which is the group of symmetries of the cube. 


\section{The Hypercubic Group $H(4)$}

The generalization from $H(3)\rightarrow H(4)$ is not too difficult to make, but there are a few subtleties that make this 
group's structure slightly harder to unravel. In 3 dimensions, reflection is not a rotation, and thus the structure of the 
total group $H(3)$ is a direct product of the group of proper symmetries, $H(3)^+$, with the group of reflections, 
$\mathbb Z / 2\mathbb Z$:
\begin{equation}
	H(3) = SH(3)\times (\mathbb Z / 2\mathbb Z)
\end{equation}
In 4 dimensions, reflection $R = diag(-1, -1, -1, -1)$ is a proper rotation, and thus there is no decomposition like this. There 
are still improper symmetries in $H(4)$, and those are characterized to have negative determinant, for example spatial 
inversion $diag(1, -1, -1, -1)$. So, $H(4)$ still has a strict subgroup of proper transformation $SH(4)$, but there is 
no direct or semidirect product structure relating the two. In fact, $H(4)^+$ is not even a quotient of $H(4)$, as can 
be seen directly by examining the orders of the two groups. 



\section{Representations of $H(4)$}

\subsection{Young Diagrams}

\end{document}