\documentclass[11pt, oneside]{article}   	% use "amsart" instead of "article" for AMSLaTeX format
\usepackage[margin = 1in]{geometry}                		% See geometry.pdf to learn the layout options. There are lots.
\geometry{letterpaper}                   		% ... or a4paper or a5paper or ... 
%\geometry{landscape}                		% Activate for rotated page geometry
%\usepackage[parfill]{parskip}    		% Activate to begin paragraphs with an empty line rather than an indent
\usepackage{graphicx}				% Use pdf, png, jpg, or eps§ with pdflatex; use eps in DVI mode
								% TeX will automatically convert eps --> pdf in pdflatex		
\usepackage{amssymb}
\usepackage{amsmath}
\usepackage[shortlabels]{enumitem}
\usepackage{float}
\usepackage{subcaption}
\usepackage{tikz-cd}
\usepackage{tikz}

\usepackage{amsthm}
\theoremstyle{definition}
\newtheorem{definition}{Definition}[section]
\newtheorem{theorem}{Theorem}[section]
\newtheorem{corollary}{Corollary}[theorem]
\newtheorem{lemma}[theorem]{Lemma}
\newtheorem{prop}{Prop}[section]

\newcommand{\N}{\mathbb{N}}
\newcommand{\R}{\mathbb{R}}
\newcommand{\Z}{\mathbb{Z}}
\newcommand{\Q}{\mathbb{Q}}

% make arrow superscripts
\DeclareFontFamily{OMS}{oasy}{\skewchar\font48 }
\DeclareFontShape{OMS}{oasy}{m}{n}{%
         <-5.5> oasy5     <5.5-6.5> oasy6
      <6.5-7.5> oasy7     <7.5-8.5> oasy8
      <8.5-9.5> oasy9     <9.5->  oasy10
      }{}
\DeclareFontShape{OMS}{oasy}{b}{n}{%
       <-6> oabsy5
      <6-8> oabsy7
      <8->  oabsy10
      }{}
\DeclareSymbolFont{oasy}{OMS}{oasy}{m}{n}
\SetSymbolFont{oasy}{bold}{OMS}{oasy}{b}{n}

\DeclareMathSymbol{\smallleftarrow}     {\mathrel}{oasy}{"20}
\DeclareMathSymbol{\smallrightarrow}    {\mathrel}{oasy}{"21}
\DeclareMathSymbol{\smallleftrightarrow}{\mathrel}{oasy}{"24}
%\newcommand{\cev}[1]{\reflectbox{\ensuremath{\vec{\reflectbox{\ensuremath{#1}}}}}}
\newcommand{\vecc}[1]{\overset{\scriptscriptstyle\smallrightarrow}{#1}}
\newcommand{\cev}[1]{\overset{\scriptscriptstyle\smallleftarrow}{#1}}
\newcommand{\cevvec}[1]{\overset{\scriptscriptstyle\smallleftrightarrow}{#1}}

%SetFonts


\title{Lattice Gauge Theory}
\author{Patrick Oare}
\date{}							% Activate to display a given date or no date

\begin{document}
\maketitle

\section{Introduction}

The essential idea of lattice gauge theory is to numerically evaluate the path integral for a quantum theory in order to determine 
correlation functions. From these correlation functions, we can get the physics. 

\section{Basic Definitions}

Instead of working with gauge connections $A_\mu(x)$, the fundamental gauge fields that we work with are the parallel 
transporters $U(x, y)$. When quantized, we call these fields $U_\mu(n)$ the \textbf{link fields}. Under a gauge transformation 
$\Omega(n)$, the fields transform as:
\begin{align}
	\psi(n)&\mapsto \Omega(n)\psi(n) \\
	U_\mu(n)&\mapsto \Omega(n) U_\mu(n) \Omega(n + \hat\mu)^\dagger
\end{align}

This allows a nice definition of the \textbf{gauge covariant derivative}. We consider both the forward difference and the 
backwards differences:
\begin{align}
	\vecc D\psi(n) &= \frac{1}{2a}\left(U_\mu(n)\psi(n + \hat\mu) - U_\mu(n - \hat\mu)^\dagger\psi(n - \hat\mu)\right) \\
	\overline\psi(n) \cev{D}&= \frac{1}{2a}\left(\overline\psi(n + \hat\mu)U_\mu(n)^\dagger - \overline\psi(n - \hat\mu)U_\mu(n - 
	\hat\mu) \right)
\end{align}
We generally will consider the difference between these two operators:
\begin{equation}
	\cevvec{D} := \vecc D - \cev D
\end{equation}

\subsection{Translational invariance on the lattice}

Propagators on the lattice are translationally invariant in the infinite volume limit. What that means is that if I want to compute 
something like:
\begin{equation}
	S_1(p) = \frac{1}{V}\sum_{x, y}e^{-ip\cdot (x - y)} S(x, y)
\end{equation}
I can choose an origin for the sum over $y$. This eliminates the sum and will give the same results with infinite statistics, 
but for practical calculations will make the results noisier. So, I can choose $y$ to be at the point $0$, which will give me:
\begin{equation}
	S_2(p) = \frac{1}{V}\sum_x e^{-ipx} S(x, 0)
\end{equation}
When evaluating $S_1(p)$ and $S_2(p)$ on the same configuration, the result will come out to be different. However, when 
we evaluate these on different configurations, they should give the same signal, albeit $S_2(p)$ will be noisier. 

\section{Lattice Units}

\section{Clover Improvement}

\section{QLUA Snippets}

\end{document}