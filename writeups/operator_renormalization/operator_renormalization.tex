\documentclass[11pt, oneside]{article}   	% use "amsart" instead of "article" for AMSLaTeX format
\usepackage[margin = 1in]{geometry}                		% See geometry.pdf to learn the layout options. There are lots.
\geometry{letterpaper}                   		% ... or a4paper or a5paper or ... 
%\geometry{landscape}                		% Activate for rotated page geometry
%\usepackage[parfill]{parskip}    		% Activate to begin paragraphs with an empty line rather than an indent
\usepackage{graphicx}				% Use pdf, png, jpg, or eps§ with pdflatex; use eps in DVI mode
								% TeX will automatically convert eps --> pdf in pdflatex		
\usepackage{amssymb}
\usepackage{amsmath}
\usepackage[shortlabels]{enumitem}
\usepackage{float}
\usepackage{tikz-cd}
\usepackage[compat=1.0.0]{tikz-feynman}   %note you need to compile this in LuaLaTeX for diagrams to render correctly
\usepackage{slashed}
\usepackage{simpler-wick}
\usepackage{subcaption}

\usepackage{amsthm}
\theoremstyle{definition}
\newtheorem{definition}{Definition}[section]
\newtheorem{theorem}{Theorem}[section]
\newtheorem{corollary}{Corollary}[theorem]
\newtheorem{lemma}[theorem]{Lemma}

\newcommand{\N}{\mathbb{N}}
\newcommand{\R}{\mathbb{R}}
\newcommand{\Z}{\mathbb{Z}}
\newcommand{\Q}{\mathbb{Q}}

% make arrow superscripts
\DeclareFontFamily{OMS}{oasy}{\skewchar\font48 }
\DeclareFontShape{OMS}{oasy}{m}{n}{%
         <-5.5> oasy5     <5.5-6.5> oasy6
      <6.5-7.5> oasy7     <7.5-8.5> oasy8
      <8.5-9.5> oasy9     <9.5->  oasy10
      }{}
\DeclareFontShape{OMS}{oasy}{b}{n}{%
       <-6> oabsy5
      <6-8> oabsy7
      <8->  oabsy10
      }{}
\DeclareSymbolFont{oasy}{OMS}{oasy}{m}{n}
\SetSymbolFont{oasy}{bold}{OMS}{oasy}{b}{n}

\DeclareMathSymbol{\smallleftarrow}     {\mathrel}{oasy}{"20}
\DeclareMathSymbol{\smallrightarrow}    {\mathrel}{oasy}{"21}
\DeclareMathSymbol{\smallleftrightarrow}{\mathrel}{oasy}{"24}
%\newcommand{\cev}[1]{\reflectbox{\ensuremath{\vec{\reflectbox{\ensuremath{#1}}}}}}
\newcommand{\vecc}[1]{\overset{\scriptscriptstyle\smallrightarrow}{#1}}
\newcommand{\cev}[1]{\overset{\scriptscriptstyle\smallleftarrow}{#1}}
\newcommand{\cevvec}[1]{\overset{\scriptscriptstyle\smallleftrightarrow}{#1}}

\newcommand{\dbar}{d\hspace*{-0.08em}\bar{}\hspace*{0.1em}}
%SetFonts




\title{Operator Renormalization}
\author{Patrick Oare}
\date{}							% Activate to display a given date or no date

\begin{document}
\maketitle

\section{Local Operator Renormalization}

Renormalization of a theory is often seen as a program that replaces the bare couplings and fields in a Lagrangian by 
renormalized quantities to make the observable quantities we can compute from a theory finite in the UV or the IR. 
However, renormalization can be viewed from a much more general perspective in terms of local composite operators. 
A \textbf{local composite operator} is an operator $\mathcal O(x)$ which is built up from the fields and derivatives in a theory 
at a single spacetime point $x$. Schematically for a theory with fermion fields $\{\Psi_i(x)\}$ and a gauge field $A_\mu$, 
such an operator looks like:
\begin{equation}
	\mathcal O_{\mu ...\nu \rho ... \sigma}(x) = \mathcal Z_{\mathcal O} D_{\mu}...D_{\nu} A_{\sigma} ... A_{\rho} 
	\Gamma_{\ell_1} ... \Gamma_{\ell_k} \Psi_{i_1} ... \Psi_{i_k} \overline \Psi_{j_1} ... \overline\Psi_{j_k}
\end{equation}
where $\mathcal Z_{\mathcal O}$ is a normalization constant which we will use to renormalize the operator $\mathcal O$, 
and $\Gamma$ is any matrix with a Dirac structure. The operator can have any number of spin indices that one wants, but 
often we will consider a relatively small value for its spin because the operators which are dominant in the OPE 
are those with the smallest twist\footnote{The twist of an operator is its spin $s$ (number of Lorentz indices) minus the mass 
dimension $d$. Operators must have twist $\geq 2$, so we will often consider twist 2 operators.}. 

Even after renormalizing all the fields and couplings in the Lagrangian, matrix elements of composite operators are often 
divergent, which is why we need to renormalize them separately. Take for example the renormalization of the operator 
$\mathcal O := \phi^2$ in the $\phi^4$ scalar field theory. When we renormalize the bare field $\phi_0 = \mathcal Z_\phi(\mu)
\phi(\mu)$, this does not immediately renormalize $\mathcal O$. This is because as an operator, $\phi^2$ is implicitly the 
normal ordering $:\phi^2:$ of the actual $\phi^2$ operator. Because of this normal ordering, renormalization constants 
are not multiplicative\footnote{More on that here: https://www.physicsoverflow.org/27963/renormalization-determined-renormalization-elementary}, i.e. we cannot just take $\mathcal O = \mathcal Z(\mu)^2$. 

To renormalize a composite operator, it suffices to compute a (divergent) Green's function and impose renormalization 
conditions on that. Typically the renormalization condition on this Green's function will be setting it equal to its 
tree level value. 

When we study composite operator renormalization, we do not need to even insert the operator into the Lagrangian. Instead, 
we will view the operator $\mathcal O$ we wish to renormalize as an \textbf{external operator} which is separate from the 
theory's Lagrangian. The goal of renormalizing $\mathcal O$ is to make any Green's function of $\mathcal O$ with another 
operator finite, and we can do this without introducing a coupling. Later we will see what happens when we renormalize 
an external operator then add it to a Lagrangian; based on the operator renormalization, we will immediately be able 
to read off how its Wilson coefficient flows. 

For any of these computations, we will need to understand how to diagrammatically compute Green's functions. We can use 
what Schwartz calls \textbf{off-shell Feynman rules} to diagrammatically compute this Green's function. Off-shell Feynman 
rules are the same as the typical Feynman rules for a theory, with some extra caveats:
\begin{itemize}
	\item We replace external polarizations with propagators. Because of this, we will often want to consider amputated 
	Green's functions without the propagators as the objects to renormalize. 
	\item Current insertions have their own set of rules, and are taken at non-zero momenta. This means currents 
	inject momentum into our diagrams, and this must be considered when labeling all the momenta. 
	\item We do not always ``work backwards" from the tip of the arrow to its bottom when computing diagrams with 
	fermion lines (as we will see in the example). 
\end{itemize}

The main tool we will use to expand these diagrams in perturbation theory is the following relation between operators  in 
the interacting theory and operators in the free theory:
\begin{equation}
	\langle 0 | T\left\{\mathcal O_1(x_1) ...\mathcal O_n(x_n)\right\} | 0 \rangle = \frac{1}{\mathcal A}\langle 0 | 
	T\left\{\exp\left[i\int d^4z\, \mathcal L_{int}^0(z)\right] \mathcal O_1^0(x_1) ... \mathcal O_n^0(x_n))\right\} | 0\rangle~
	\label{eq:free_expansion}
\end{equation}
where $\mathcal O^0$ denotes an operator in the free theory, and $\mathcal L_{int}^0$ is the interaction Lagrangian 
in the free theory. $\mathcal A$ is a normalization given by:
\begin{equation}
	\mathcal A = \langle 0 | T \left\{\exp\left(i\int d^4 z\,\mathcal L_{int}^0(z)\right)\right\} | 0\rangle
\end{equation}
With this expansion, we can draw Feynman diagrams. Intuitively it helps to compare this to a derivation of Feynman 
diagrams in Schwartz, where he shows how to compute $n$-point functions diagrammatically using Equation 7.64 to 
relate them to their free field values:
\begin{equation}
	\langle 0 | T\left\{\phi(x_1) ... \phi(x_n)\right\} | 0\rangle = \frac{1}{\mathcal A} \langle 0 | T\left\{\exp\left[i\int d^4z\, \mathcal 
	L_{int}^0(z)\right]\phi^0(x_1) ... \phi^0(x_n)\right\} | 0\rangle
\end{equation}
The key point in this equation (as in Equation~\ref{eq:free_expansion}) is that we can Taylor expand the exponential and 
contract the fields on the right hand side of the equation to give ourselves propagators and derive Feynman rules for the 
theory. 

Once we renormalize an operator, an important quantity to compute is the operator's \textbf{anomalous dimension} 
$\gamma_\mathcal{O}$, which describes how the operator flows under renormalization:
\begin{equation}
	\gamma_\mathcal{O} := -\frac{\mu}{Z_\mathcal{O}}\frac{d \mathcal{Z_{\mathcal O}}}{d\mu} = -\mu\frac{d}{d\mu}\log
	\mathcal Z_\mathcal{O}
\end{equation}
Once we have determined the counterterm for the operator $\mathcal{Z_\mathcal{O}}$, we can immediately get the anomalous 
dimension. This quantity acts similarly to the $\beta$ function for a coupling, and also appears in the Callan-Symanzik 
equation on equal terms. 

TODO discuss Callan-Symanzik.

\subsection{Example: QED current}

We will do an example in QED, and compute the renormalization of $j^\mu(x) = \overline \psi(x)\gamma^\mu \psi(x)$. 
The simplest non-vanishing Green's function involving $j^\mu(x)$ is the three point function, which we will expand 
in momentum space:
\begin{equation}
	\langle j^\mu(x) \psi(x_1) \overline\psi(x_2)\rangle = \int \dbar^4 p\,\dbar^4 q_1\,\dbar^4 q_2\, e^{-ipx}e^{-ip_1 x_1}
	e^{iq_2x_2} i\mathcal M^\mu(p, q_1, q_2) (2\pi)^4 \delta^4(p + q_1 - q_2)
\end{equation}
We have chosen the signs on the momenta to replicate a current insertion of momentum $p$; namely, if a particle is 
propagating initially with momentum $q_1$, the current insertion knocks the momentum of the particle so that:
\begin{equation}
	q_2 = p + q_1
\end{equation}

We will compute this in perturbation theory to one loop and associate a diagram with each term. Using the expansion in 
Equation~\ref{eq:free_expansion}, we can expand the exponential to one loop:
\begin{equation}
	\exp\left(i\int d^4z\,\mathcal L_{int}^0(z)\right) \sim 1 + i\int d^4 z\, \mathcal L_{int}^0 + \frac{i^2}{2} \int d^4 z\, d^4 z'\, 
	\mathcal L_{int}^0(z) \mathcal L_{int}^0(z') + ...
\end{equation}
When inserted into the equation, the first piece gives the tree level vertex and the piece with two integrals gives the first 
loop correction. Plugging in at first order, we can use Wick's theorem to evaluate the free field correlators (here $\alpha, 
\beta, \rho, \sigma$ are Dirac indices):
\begin{align}
	\langle j^\mu(x) \psi(x_1) \overline\psi(x_2)\rangle_\textnormal{tree} &= \langle j^\mu_0(x) \psi_0(x_1) 
	\overline\psi_0(x_2)\rangle \\
	&= \langle \wick{
		\c1{\overline\psi_0^\alpha}(x) \gamma^\mu_{\alpha\beta} \c2{\psi_0^\beta}(x) \c1{\psi_0^\rho}(x_1) 
		\c2{\overline \psi_0^\sigma}(x_2) 
	}\rangle \\ 
	&= S(x_1, x) \gamma^\mu S(x, x_2)
\end{align}
as $\langle \psi(x) \overline \psi(y)\rangle = S(x, y) = i\int \dbar^4 k\,\frac{e^{-ik(x - y)}}{\slashed{k} - m}$. We can take this tree 
level result to momentum space, where we see:
\begin{align}
	\mathcal M^\mu(p, q_1, q_2)_\textnormal{tree} &= \int d^4 x\,d^4 x_1\, d^4x_2\, e^{ipx}e^{iq_1 x_1} e^{-iq_2 x_2} \langle 
	j^\mu(x) \psi(x_1)\overline\psi(x_2)\rangle_{tree} \nonumber \\
	&= \int d^4 x\,d^4 x_1\, d^4x_2\, e^{ipx}e^{iq_1 x_1} e^{-iq_2 x_2} \int \dbar^4 k\, \int \dbar^4 k'\, 
	\frac{ie^{-ik(x_1 - x)}}{\slashed{k} - m} \gamma^\mu \frac{ie^{-ik'(x - x_2)}}{\slashed{k'} - m} \nonumber \\
	&= \int\dbar^4 k\,\dbar^4 k'\, d^4 x\, d^4 x_1\,d^4 x_2 e^{i(p + k - k')x}e^{i(q_1 - k)x_1}e^{i(k' - q_2)x_2} 
	\frac{i}{\slashed k - m}\gamma^\mu\frac{i}{\slashed k' - m} \nonumber\\
	&= \frac{i}{\slashed q_1 - m} \gamma^\mu\frac{i}{\slashed q_2 - m}\delta^4(p + q_1 - q_2)
\end{align}
where we leave the delta function in to emphasize that we \textbf{must enforce momentum conservation from current 
insertion}. Diagrammatically, this corresponds to the diagram
\begin{equation}
\begin{gathered}
\feynmandiagram [large, horizontal'=a to c] {
%a [particle = \(q_1\)] -- [insertion = {[size = 6pt, style = red] .5}] c [particle = \(q_2\)]
a [particle = \(q_1\)] -- [fermion] b [dot, style = red],
b -- [fermion] c [particle = \(q_2\)]
};
\end{gathered}~
\label{eq:current_insertion}
\end{equation}
where the red dot denotes a current insertion of momentum $p$. 

Now we move towards the actual process of renormalization: we can expand Equation~\ref{eq:free_expansion} to 
one loop and compute the one loop 1PI correction, then renormalize the operator. This is going to get very tiring very soon, 
but I think it is instructive to get familiar with where the Feynman rules come from for operator insertions. At one 
loop, we pick up the $(\mathcal L_{int}^0)^2$ piece in our free theory Green's function (recall $\mathcal L_{int}^0(z) = 
-e\overline\psi_0(z) \slashed A^0(z) \psi_0(z)$). This gives us (also switching the Dirac indices to be Latin indices because 
there are a lot of them and I don't know the entire Greek alphabet):
\begin{align}
	\langle j^\mu(x) \psi^a(x_1)\overline \psi^b(x_2)\rangle_\textnormal{1-loop} = -\frac{e^2}{2}\int d^4z\, d^4z'\,\langle 
	\overline\psi_0^c(z) 
	&\gamma^\nu_{cd} A_\nu^0(z) \psi_0^d(z)
	\times\nonumber \\
	\overline\psi_0^e(z')\gamma^\lambda_{ef} A_\lambda^0(z')\psi_0^f(z')
	&\overline\psi_0^g(x) \gamma_{gh}^\mu \psi_0^h(x) \psi_0^a (x_1) \overline \psi_0^b(x_2)\rangle 
\end{align}
We now have to sum on all contractions of each of the fields. However, we can simplify this by remembering that we are only 
dealing with connected, 1PI diagrams and throw away many of the contractions. Namely, the general structure of the correlator 
will look like this:
\begin{equation}
	\langle j^\mu(x) \psi^a(x_1)\overline \psi^b(x_2)\rangle_\textnormal{1-loop}\sim \frac{(-ie)^2}{2}\int \langle A(z) A(z')\rangle 
	\langle \overline\psi(z) \psi(z) \overline\psi(z') \psi(z') \overline\psi(x)\psi(x) \overline\psi(x_1)\psi(x_2)\rangle
	\nonumber
\end{equation}
where we sum over contractions for the fermionic correlator. Because the photon propagator $S^{(\gamma)}_{\nu\lambda}(z, 
z') = \langle A_\nu^0(z) A_\lambda^0(z')\rangle$ already connects the $z$ and $z'$ points, this limits the structure of this to 
two specific contractions:
\begin{equation}
	\langle j^\mu(x) \psi^a(x_1)\overline \psi^b(x_2)\rangle_\textnormal{1-loop} \sim\frac{(-ie)^2}{2}\int S^{(\gamma)}_{z, z'}
	\left[
	\langle \wick{
	\c1{\overline \psi_z} \c2{\psi_z} \c3{\overline\psi_{z'}}  \c4{\psi_{z'}} \c2{\overline\psi_x} \c3{\psi_x} \c1{\psi_{x_1}}
	\c4{\overline\psi_{x_2} }
	}\rangle + \langle \wick{
	\c1{\overline \psi_z} \c2{\psi_z} \c3{\overline\psi_{z'}}  \c4{\psi_{z'}} \c4{\overline\psi_x} \c1{\psi_x} \c3{\psi_{x_1}} 
	\c2{\overline\psi_{x_2}}
	} \rangle \right]
	\nonumber
\end{equation}
Diagrammatically, two diagrams which contribute look like this:
\begin{equation}
\begin{gathered}
\feynmandiagram [small, horizontal'=a to b] {
a [particle = \(z\)] -- [boson] b [particle = \(z'\)],
c [particle = \(x\)] -- a,
c -- b,
d [particle = \(x_1\)] -- a,
e [particle = \(x_2\)] -- b
};
\end{gathered}~
\label{eq:1pi}
\end{equation}
where the second term has the $z$ and $z'$ vertices flipped. These will end up being the same and canceling the factor of 1/2. 

We only get these two contractions for a few reasons. 
First, because of the photon propagator connecting $z$ and $z'$, we cannot contract the fermion fields at $z$ and $z'$ 
together. If we did, we would get a graph which is not 1P1, because we could cut the bubble out of the diagram. Second, 
we cannot contract the $x_1$ and $x_2$ points together, or else we would have a disconnected diagram, hence we must 
contract $x_1$ and $x_2$ to one of $z$ or $z'$ (therefore we have two diagrams). This is easiest to see pictorially-- in the 
manner of the previous diagram. 

So, we have reduced this computation to calculating (now restoring the Dirac structure):
\begin{align}
	\langle j^\mu(x) \psi^a(x_1)\overline \psi^b(x_2)\rangle_\textnormal{1-loop} &= (-ie)^2 \int d^4z\,d^4z' S^{(\gamma)}_{\nu 
	\lambda}(z, z') \langle \wick{
	\c1{\overline \psi_z^c} \gamma^\nu_{cd} \c2{\psi_z^d} \c3{\overline\psi^e_{z'}} \gamma^\lambda_{ef} \c4{\psi_{z'}^f} 
	\c2{\overline\psi_x^g} \gamma^\mu_{gh} \c3{\psi_x^h} \c1{\psi_{x_1}^a}
	\c4{\overline\psi_{x_2}^b}
	}\rangle \nonumber \\
	&= (-ie)^2 \int d^4 z\, d^4 z' S^{(\gamma)}_{\nu\lambda}(z, z') S^{ac}(x_1, z) \gamma^\nu_{cd} S^{dg}(z, x) 
	\gamma^\lambda_{ef}S^{he}(x, z') \gamma_{gh}^\mu S_{fb}(z', x_2) \nonumber \\
	&= (-ie)^2 \int d^4 z \int d^4 z' S_{\nu\lambda}^{(\gamma)}(z, z') S(x_1, z)\gamma^\nu S(z, x) \gamma^\mu S(x, z') 
	\gamma^\lambda S(z', x_2)
\end{align}
where:
\begin{equation}
	S^{(\gamma)}_{\mu\nu}(x, y) = \int d^4k\frac{-ie^{ik(x - y)}g_{\mu\nu}}{k^2 + i\epsilon}
\end{equation}
is the photon propagator. We can now plug this into our expression for $\mathcal M^\mu(p, q_1, q_2)$ and associate a 
one-loop diagram to this computation (where the measure $d^{20}x = d^4x\,d^4x_1\,d^4 x_2\,d^4z\,d^4z'$):
\begin{align}
	\mathcal M^\mu&(p, q_1, q_2)_\textnormal{1-loop} = (-ie)^2\int d^{20} x\,e^{i(px + q_1 x_1 - q_2 x_2)} 
	S_{\nu\lambda}^{(\gamma)}(z, z') S(x_1, z)\gamma^\nu S(z, x) \gamma^\mu S(x, z') \gamma^\lambda S(z', x_2) 
	\nonumber \\
	&= (-ie)^2 \int d^{20}x \dbar^{20}k\, e^{i(px + q_1 x_1 - q_2 x_2)}\frac{-ie^{ik(z - z')}}{k^2}
	\frac{ie^{-ik_1(x_1 - z)}}{\slashed k_1 - m}\gamma^\nu \frac{ie^{-ik_2(z - x)}}{\slashed k_2 - m} \gamma^\mu \frac{ie^{-ik_3 
	(x - z')}}{\slashed k_3 - m} \gamma_\nu \frac{ie^{-ik_4 (z' - x_2)}}{\slashed k_4 - m} \nonumber \\
	&= (-ie)^2 \int d^{20}x \dbar^{20} k\, e^{i(p + k_2 - k_3)x}e^{i(k + k_1 - k_2)z} e^{i(k_3 - k_4 - k)z'}e^{i(q_1 - k_1)x_1}
	e^{-i(q_2 - k_4)x_2} \times \nonumber \\
	&\;\;\;\;\;\;\;\;\;\;\;\;\;\;\;\;\;\;\;\;\;\;\;\; \left(\frac{-i}{k^2} \frac{i}{\slashed k_1 - m} \gamma^\nu \frac{i}{\slashed k_2 - m}
	\gamma^\mu\frac{i}{\slashed k_3 - m} \gamma_\nu \frac{i}{\slashed k_4 - m}\right) \nonumber \\
	&= (-ie)^2 \frac{i}{\slashed q_1 - m}\left[\int\dbar^4 k \frac{i}{k^2}\gamma^\nu \frac{i}{\slashed q_1 + \slashed k - m}
	\gamma^\mu \frac{i}{\slashed q_2 + \slashed k - m}\gamma_\nu\right]\frac{i}{\slashed q_2 - m}
\end{align}
The delta functions enforce momentum conservation at each vertex in the following diagram, and we can thus 
encode this amplitude using the off-shell Feynman rules on the following diagram, where the current insertion 
injects momentum $p$ into the diagram, as well as a $\gamma^\mu$ factor. Our result is thus:
\begin{align}
	i\mathcal M^\mu(p, q_1, q_2)_\textnormal{1-loop} &= \begin{gathered}
	\feynmandiagram[large, horizontal' = a to b] {
	a -- [boson, reversed momentum' = \(k\)]b,
	%a -- [fermion, half left, momentum = \(k + q_1\), insertion = {[size = 6pt, style = red] .5}] b,
	a -- [fermion, momentum = \(q_1 + k\)] e [dot, style = red],
	e -- [fermion, momentum = \(q_2 + k\)] b,
	c -- [fermion, momentum = \(q_1\)] a,
	b -- [fermion, momentum = \(q_2\)] d 
	};
	\end{gathered} \\
	&= (-ie)^2 \frac{i}{\slashed q_1 - m}\left[\int\dbar^4 k \frac{i}{k^2}\gamma^\nu \frac{i}{\slashed q_1 + \slashed k - m}
	\gamma^\mu \frac{i}{\slashed q_2 + \slashed k - m}\gamma_\nu\right]\frac{i}{\slashed q_2 - m}
\end{align}
where $q_2 = p + q_1$ is the result of current insertion at the red dot in the diagram. At this point, we can evaluate this loop 
integral in dimensional regularization. Stripping off the mass dimensions of the electron coupling with 
$e\mapsto \mu^{\epsilon}e$ in $d = 4 - \epsilon$ dimensions, we have:
\begin{equation}
	i\mathcal M^\mu(p, q_1, q_2) = i\mathcal M^\mu_\textnormal{tree}(p, q_1, q_2) \frac{e^2}{16\pi^2}\frac{2}{\epsilon}
\end{equation}

To finish our renormalization process, we must include the counterterm diagram as well by expanding $\mathcal Z_j = 1 + 
\delta_j$ in the Green's function. We have:
\begin{align}
	\delta_j \mathcal M^\mu(p, q_1, q_2) &= 
	\begin{gathered}
        \feynmandiagram [large, horizontal'=a to c] {
        a [particle = \(q_1\)] -- [fermion] b [crossed dot, style = red],
b -- [fermion] c [particle = \(q_2\)]
};
\end{gathered} \\
&= i\mathcal M^\mu_\textnormal{tree}(p, q_1, q_2) \delta_j
\end{align}
where the red $\otimes$ denotes a counterterm insertion. We can thus read off the counterterm:
\begin{equation}
	\delta_j = -\frac{e^2}{16\pi^2}\frac{2}{\epsilon}
\end{equation}

Now that we have read off the counterterm, the bulk of the work is done! Notice that this calculation is much easier to do 
diagrammatically with Feynman rules than from first principles; however, I wanted to illustrate how to do build up and 
motivate Feynman rules for operator insertions, i.e. the off-shell Feynman rules that Schwartz discusses. Note that in 
this case, we read the fermion arrow from backwards to forwards because of the position of the propagators. In general, 
I would suggest doing Wick contractions at tree level first to determine the order of the propagators in the diagram, and then 
using this to move to the loop diagrams. 

To compute the anomalous dimension, the bare current can be expressed as:
\begin{equation}
	J_{bare}^\mu = \overline\psi_0\gamma^\mu\psi_0 = \mathcal Z_J^{-1}\mathcal Z_\psi J_R^\mu
\end{equation}
where $\psi_0$ is the bare field. Since this is purely a function of the bare parameters, it cannot run with $\mu$. Using 
$\mathcal Z_J = \mathcal Z_\psi$, we can differentiate this to find:
\begin{equation}
	\mu \frac{dJ^\mu}{d\mu} = 0
\end{equation}
where $J^\mu$ is the renormalized current. 

% TODO study amputated Green's functions



\section{Regularization-Independent Momentum Subtraction (RI-MOM)}

The RI-MOM scheme (also known as Rome-Southampton) is a renormalization scheme well equipped to deal with 
the lattice, namely because we can calculate the relevant quantities that define the scheme easily on the lattice. The 
lattice spacing $a$ provides us a with a natural UV regulator, and RI-MOM will tell us how to go from such a regulated 
result to a physical observable quantity. 

The RI-MOM has a relatively simple renormalization condition. We will define it here for an arbitrary Green's function, 
for example a three point function. For a renormalization scale $\mu$ and working in a fixed gauge, \textbf{we 
define the amputated, renormalized Green's function at momentum $p^2 = -\mu^2$ to be equal to its tree level 
value}. 

We will denote bare quantities as $a^{(0)}$, and renormalized quantities with a $R$ subscript, when appropriate. 
Note that any quantities computed directly on the lattice are bare, and the entire point of using RI-MOM is to extract 
a sensible definition for these bare quantities. 

In practice on the lattice, there are two things that we must compute. We will work with a specific example here for a 
given quark field $q(x)$. Suppose the operator we are trying to compute is $\mathcal O(z)$. We will renormalize the 
three point function:
\begin{equation}
	G(p) = \frac{1}{V}\sum_{x, y, z}e^{ip\cdot (x - y)}\langle q(x)\mathcal O(z) \overline{q}(y)\rangle~
	\label{eq:green}
\end{equation}
i.e. we are projecting the source and sink to a definite momentum and projecting the operator $\mathcal O(z)$ to zero 
momentum. 

We also will need to compute the momentum projected propagator:
\begin{equation}
	S(p) = \frac{1}{V}\sum_{x, y}e^{ip\cdot(x - y)} S(x, y)
\end{equation}
where $S^{ab}_{ij}(x, y) = \langle q^a_i(x)\bar q^b_j(y)\rangle$ is the standard propagator. On the lattice, the two objects that 
we must compute directly are $S(p)$ and $G(p)$, and everything else follows once we have these quantities. 

To explain the method, our goal is to compute the operator renormalization:
\begin{equation}
	\mathcal O_R(\mu) = \mathcal Z(\mu) \,\mathcal O^{(0)}
\end{equation}
where $\mathcal O_R(\mu)$ is our renormalized operator, $\mathcal Z(\mu)$ is the renormalization coefficient of 
interest, and $\mathcal O_{lat}$ is the lattice (bare) operator. We will also assume the quark fields have been 
renormalized by some quark field renormalization $\mathcal Z_q$ (note this is the opposite convention studied in 
many QFT classes):
\begin{equation}
	q_R(\mu) = \sqrt{\mathcal Z_q}\,q^{(0)}
\end{equation}
There is an analytical expression for $\mathcal{Z}_q$ on the lattice in the RI-MOM scheme, and it has been determined 
to be:
\begin{equation}
	\mathcal Z_q(p)|_{p^2 = -\mu_R^2} = \left[\frac{tr\left\{-i\sum_{\nu = 1}^4 \gamma_\nu \sin(ap_\nu) a S(p)^{-1}\right\}}{12\sum_{\nu = 1}^4 \sin^2(ap_\nu)}\right]_{p^2 = -\mu^2}
\end{equation}
The twelve on the bottom is a normalization $12 = 3\times 4$ for the number of color and number of spin indices, which we will 
see in many of the expressions. 

Let $\Gamma(p)$ be the \textbf{amputated three point function}, bare or renormalized. We can relate $\Gamma$ to the other 
quantities we have already computed by using the inverse propagator to manually cut the legs off the full three 
point function:
\begin{equation}
	\Gamma(p) = S(p)^{-1} G(p) S(p)^{-1}
\end{equation}
We will denote the tree level version of this by $\Gamma_B$, where the $B$ subscript stands for ``Born". 

Using our quantities already computed on the lattice, we can compute the bare $\Gamma^{(0)}(p)$ directly in terms 
of the renormalized quantities. In the continuum limit, the renormalized Green's function will be:
\begin{align}
	G_R(p; \mu) &= \int d^4x \, d^4 y\, d^4 z\, e^{ip\cdot (x - y)} \langle q_R(x; \mu) \mathcal O_R(z; \mu) \overline 
	q_R(y; \mu) 
	\rangle \\
	&= \mathcal Z_q(\mu) \mathcal Z(\mu) \int d^4x \, d^4 y\, d^4 z\, e^{ip\cdot (x - y)} \langle q^{(0)}(x) \mathcal O^{(0)}
	(z) q^{(0)}(y) \rangle \\
	&= \mathcal Z_q(\mu) \mathcal Z(\mu) G^{(0)}(p)
\end{align}
Similarly, the bare and renormalized propagators are related as:
\begin{equation}
	S_R(p; \mu) = \int d^4x d^4 y\, e^{ip\cdot(x - y)}\langle q_R(x; \mu)\overline q_R(y; \mu)\rangle = \mathcal Z_q(\mu) 
	S^{(0)}(p)
\end{equation}
These relation translates immediately to the amputated Green's function $\Gamma(p)$, as $S^{-1}_R(p; \mu) = \mathcal 
Z_q^{-1}(\mu) (S^{(0)})^{-1}$ we see that:
\begin{equation}
	\Gamma_R(p; \mu) = \mathcal Z_q(\mu)^{-1}\mathcal Z(\mu) \Gamma^{(0)}(p)
\end{equation}

We are now in a position to apply the renormalization condition. We must equate the renormalized, amputated Green's function 
to the tree level Green's function $\Gamma_B(p)$. This gives us:
\begin{equation}
	\mathcal Z_q(\mu)^{-1}\mathcal Z(\mu) \Gamma(p) = \Gamma_B(p)
\end{equation}
Dividing by a conventional factor of 12 as a normalization and inverting, we can clean this expression up into a simple equation 
for $\mathcal Z(\mu)$:
\begin{equation}
	\mathcal Z(p^2 = -\mu^2) = \left[\frac{12\mathcal Z_q(p)}{tr\left\{\Gamma(p)\Gamma_B(p)^{-1}\right\}}\right]_{p^2 
	= -\mu^2}
\end{equation}

\section{Matching to a continuum scheme}

\section{Example: Isospin}

We are interested here in computing matrix elements of the operator:
\begin{equation}
	\mathcal O(z) = \mathcal O_u(z) - \mathcal O_d(z)~
	\label{eq:operator_dfn}
\end{equation}
where the quark operators $\mathcal O_q$ are given by
\begin{equation}
	\mathcal O_q(z) = \frac{1}{\sqrt{2}}(\mathcal T^q_{33}(z) - \mathcal T^q_{44}(z))
\end{equation}
and the irreducible tensor operators $\mathcal T^q_{\mu\nu}$ are defined as
\begin{equation}
	\mathcal T^q_{\mu\nu} = \overline q(z)\, \gamma_{\{\mu} \cevvec{D}_{\nu\}}\, q(z)
\end{equation}
with $\cevvec D = \vec D - \cev D$ the symmetrized covariant derivative. Note we define the symmetric and traceless 
component of a tensor to be:
\begin{equation}
	a_{\{\mu}b_{\nu\}} = \frac{1}{2}(a_\mu b_\nu + a_\nu b_\mu) - \frac{1}{4}a_\alpha b^\alpha g_{\mu\nu}
\end{equation} 

We are inserting the operator $\mathcal O_q$ with momentum $\vec p = 0$. We will focus our analysis on the tensor operator 
$\mathcal T^q_{\mu\mu}$ (note there is no sum on $\mu$ here), and note that $\mathcal O(z)$ can be obtained through 
linearity. We may write:
\begin{equation}
	\sum_z\mathcal T^q_{\mu\mu}(z) = \sum_{z, z'}\overline q(z)\, J_\mu(z, z')\,q(z')~
	\label{eq:operator_mom_proj}
\end{equation}
Plugging in the definition of the derivatives:
\begin{align}
	\vec D\psi(z) &= \frac{1}{2}\left(U_\mu(z)\psi(z + \hat\mu) - U_\mu(n - \hat\mu)^\dagger \psi(z - \hat\mu)\right) \\
	\overline\psi(z) \cev D &= \frac{1}{2}\left(\overline\psi(z + \hat\mu)U_\mu(z)^\dagger - 
	\overline\psi(z - \hat\mu) U_\mu(z - \hat\mu)\right)
\end{align}
we find the current $J_\mu(z, z')$ is:
\begin{equation}
	J_\mu(z, z') = \left[U_\mu(z) \delta_{z + \hat\mu, z'} - U_\mu(z')^\dagger\delta_{z - \hat\mu, z'}\right]\gamma_\mu
\end{equation}

We may now use this expansion to compute the three point function for the operator $\mathcal T_\mu = 
\mathcal T^u_{\mu\mu} - \mathcal T^d_{\mu\mu}$ (we can simply take $\mathcal T_3 - \mathcal T_4$ to get the operator of 
interest in Equation~\ref{eq:operator_dfn}). Using Equation~\ref{eq:operator_mom_proj}, we write 
\begin{equation}
	\sum_z \mathcal T_\mu(z) = \sum_{z, z'}\left[\overline u(z)\, J_\mu(z, z')\, u(z') - \overline d(z)\, J_\mu(z, z')\, d(z')\right]
\end{equation}
Plugging this into Equation~\ref{eq:green}, we find that we can expand the total up quark Green's function (here $\alpha, 
\beta$ are Dirac indices) as:
\begin{equation}
	G^{\alpha\beta}(p) = \frac{1}{\sqrt 2}\left(G_3^{\alpha\beta}(p) - G_4^{\alpha\beta} (p)\right)
\end{equation}
where:
\begin{align}
	G_\mu^{\alpha\beta}(p) &= \frac{1}{V}\sum_{x, y, z} e^{-ip(x - y)}\langle u^\alpha(x)\mathcal T_\mu(z) \overline u^\beta(y)
	\rangle \\
	&= \frac{1}{V}\sum_{x, y, z, z'} e^{-ip(x - y)}\left[\langle u^\alpha(x)\overline u^\sigma(z) J_\mu^{\sigma\rho}(z, z') 
	u(z')^\rho\overline u^\beta(y)\rangle - \langle u^\alpha(x)\overline d^\sigma(z) J_\mu^{\sigma\rho}(z, z') d^\rho(z') 
	\overline u^\beta(y)\rangle 
	\right]
\end{align}
Now we perform all possible Wick contractions on the matrix elements to write them as propagators:
\begin{align}
	\langle u^\alpha(x)\overline u^\sigma(z) &J_\mu^{\sigma\rho}(z, z') u^\rho(z')\overline u^\beta(y)\rangle = 
	\langle \wick{
		\c1 u \c1{\overline u} J \c2 u \c2{\overline u}
	} \rangle + \langle \wick{
		\c1 u \c2{\overline u} J \c2 u \c1{\overline u} 
	} \rangle \nonumber \\ 
	&= S^{\alpha\sigma}(x, z) J^{\sigma\rho}_\mu(z, z') S^{\rho\beta}(z', y) + (-1)^3 S^{\alpha\beta}(x, y) 
	J^{\sigma\rho}_\mu(z, z')S^{\rho\sigma}(z', z) \\
	\nonumber\\
	\langle u^\alpha(x)\overline d^\sigma(z) &J_\mu^{\sigma\rho}(z, z') d^\rho(z') \overline u^\beta(y)\rangle = 
	\langle \wick{
		\c1 u \c2{\overline d} J \c2 d \c1{\overline u}
	}\rangle\nonumber \\ 
	&= (-1)^3 S^{\alpha\beta}(x, y) J^{\sigma\rho}_\mu(z, z')S^{\rho\sigma}(z', z)
\end{align}
where the factors of $(-1)$ come from rearranging the contraction so that the contracted pieces are of the form 
$\langle u\overline u\rangle$. The vacuum pieces cancel because the up and down quark propagators are degenerate, so the 
final result is very clean:
\begin{equation}
	G_\mu(p) = \frac{1}{V}\sum_{x, y, z, z'}e^{ip(x - y)} S(x, z) J_\mu (z, z') S(z', y)~
	\label{eq:greens_function}
\end{equation}
This is our central equation, but note that initially there is a difficulty with 

There are two primary ways to compute this on the lattice.
We can compute this directly using momentum sources, or 
we can use the sequential source technique. Momentum sources work specifically for Equation~\ref{eq:greens_function}, 
but produce a significantly better signal on a small number of configurations. On the other hand, sequential source 
is much more general, but produces more noise. We will discuss each method below.
%either compute this \textbf{through the operator} or 
%\textbf{through the sink}, and below we will discuss each method and the pros and cons of each. 

\subsection{Momentum sources}

Observe that we can rewrite Equation~\ref{eq:greens_function} as:
\begin{align}
	G_\mu(p) &= \frac{1}{V}\sum_{x, y, z, z'} e^{ipx} S(x, z) J_\mu(z, z') e^{-ipy} S(z', y) \nonumber\\
	&= \frac{1}{V}\sum_{z, z'}\gamma_5\left(\sum_x S(z, x) e^{ipx}\right)^\dagger\gamma_5 J_\mu(z, z') \left(\sum_y S(z', y) 
	e^{ipy} \right) \nonumber\\
	&= \frac{1}{V}\sum_{z, z'} \gamma_5 \tilde S_p(z)^\dagger \gamma_5 J_\mu(z, z') \tilde S_p(z')~
	\label{eq:through_sink}
\end{align}
where we have defined $\tilde S_p(z)$ as:
\begin{equation}
	\tilde S_p(z) = \sum_x S(z, x) e^{ipx}
\end{equation}

The advantage of casting the equation in this form is that we can solve for $\tilde S_p(z)$ directly by inverting the Dirac 
equation with a momentum source, i.e. we have:
\begin{equation}
	\sum_{z} D(x, z) \tilde S_p(z) = e^{ipx}
\end{equation}
where $D(x, z)$ is the Dirac operator. This means that upon solving for $\tilde S_p(z)$ and plugging this into 
Equation~\ref{eq:through_sink}, we can solve directly for $G_\mu(p)$. 

This is an exact equation and it does not rely on translational invariance in the infinite statistics limit. 
Therefore, this method will give much better signal and can be run efficiently on a small number of configurations. The 
downside to this is that we require a propagator inversion for each choice of sink momentum. 
To compute $G(p)$ for a large number of sink momenta, as we need to do to extrapolate $\mathcal Z(\mu)$ in the 
continuum limit, a propagator inversion at each sink momenta is not feasible. We must instead choose the sink momentum 
wisely to be able to extract the discretization artifacts and extrapolate $\mathcal Z(\mu)$ to the continuum (we will describe 
these discretization artifacts in Section~\ref{sec:artifacts}. 

\subsection{Sequential source method}

In practice we will use the sequential source method, which if implemented correctly does not force us to invert a propagator 
at every sink momenta. This technique is also much more general than the one previously described, but it suffers from 
more noise because it relies on the translational invariance of the lattice, which only exists in the infinite statistics 
limit. The idea of the sequential source method is that if we have an equation involving the full propagator $S(x, y)$, we can 
invert a source which depends on the propagator $S(x)$. For example, in this problem we wish to evaluate 
Equation~\ref{eq:greens_function}, but we cannot simply evaluate $S(x, y)$ for every $x$ and $y$. To get around this, 
consider using a source
\begin{equation}
	b(z) = \sum_{z'} J_\mu(z, z') S(z', 0)~
	\label{eq:source}
\end{equation}
to invert the Dirac equation, which will solve for $M(x)$ in this equation:
\begin{equation}
	\sum_x D(z, x) M(x) = b(z)
\end{equation} 
where $D(x, z)$ is the Dirac operator. Upon inversion, using that $\sum_z S(y, z) D(z, x) = \delta(y - x)$, we can move the 
Dirac operator to the other side as $D^{-1}(y, z) = S(y, z)$ and obtain:
\begin{equation}
	M(x) = \sum_z D^{-1}(x, z) b(z) = \sum_{z, z'} S(x, z) J_\mu(z, z') S(z', 0)~
	\label{eq:inversion}
\end{equation}
Note that we have summed the full propagator $S(x, z)$ for the price of a single inversion of the source $b(z)$. We can then 
reconstruct Equation~\ref{eq:greens_function} in the infinite statistics limit when translational invariance is restored:
\begin{equation}
	G_\mu(p) \rightarrow \frac{1}{V} \sum_x e^{ipx} M(x) = \frac{1}{V} \sum_{x, z, z'} e^{ipx} S(x, z) J_\mu(z, z') S(z', 0)
\end{equation}

In our case with a large amount of sink momenta, this method is much more robust than inverting a momentum source 
because we one inversion can give us $G(p)$ at every value of the sink momentum. We will also call this construction going 
\textbf{through the operator}, because the inversion in Equation~\ref{eq:inversion} projects the current insertion onto $q = 0$ 
momentum. If we had been interested in projecting the operator onto different momentum values, then we would need to use a 
new sequential source (modify Equation~\ref{eq:source}) for each value of the operator momentum. Pictorially, we are 
inverting at the operator momentum, then tying up at the sink momentum. On the other hand, we reverse the direction of 
inversion and invert our propagator at each sink momentum first, then tie up the line at the operator. This method is called 
going \textbf{through the sink}. We can represent these different methods below, where in our case $q = 0$. 
\begin{figure}[H]
	\centering
	\begin{subfigure}[t]{.48\textwidth}
	\centering
		\feynmandiagram [large, horizontal'=a to c] {
		a [particle = \(p\)] -- [fermion, bend left = 60] b [dot, style = red],
		d -- [boson, momentum = \(q\)] b,
		b -- [fermion, bend left = 60] c  [particle = \(p + q\)],
		a -- [fermion, bend right = 20, opacity = 0] c
                };
            	\caption{Inversion through the operator}
	\end{subfigure}
	~
	\begin{subfigure}[t]{.48\textwidth}
	\centering
		\feynmandiagram [large, horizontal'=a to c] {
		%a [particle = \(p\)] -- [fermion, bend right] c [particle = \(p + q\)],
		%c -- [fermion, bend right] b [empty dot, style = red],
		%d -- [boson, momentum = \(q\)] b
		a [particle = \(p\)] -- [fermion, bend left = 60, opacity = 0] b [dot, style = red],
		d -- [boson, momentum = \(q\)] b,
		b -- [anti fermion, bend left = 60] c  [particle = \(p + q\)],
		a -- [fermion, bend right = 20] c
                };
            	\caption{Inversion through the sink}
	\end{subfigure}
\end{figure}

In this problem inversion through the sink would require too many propagator inversions like in the previous momentum 
source method, and it would also be noisy like inversion through the operator. As such, there is no reason to consider it, 
and I included it here mainly for generality. 

\section{Hypercubic Artifacts}
\label{sec:artifacts}

When we compute observables at a finite lattice spacing $a$, we suffer discretization artifacts which are relics of the explicit 
symmetry breaking $SO(1, 3)\rightarrow H(4)$ suffered by putting the theory on a lattice. 

\end{document}