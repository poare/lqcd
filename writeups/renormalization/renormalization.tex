\documentclass[11pt, oneside]{article}   	% use "amsart" instead of "article" for AMSLaTeX format
\usepackage[margin = 1in]{geometry}                		% See geometry.pdf to learn the layout options. There are lots.
\geometry{letterpaper}                   		% ... or a4paper or a5paper or ... 
%\geometry{landscape}                		% Activate for rotated page geometry
%\usepackage[parfill]{parskip}    		% Activate to begin paragraphs with an empty line rather than an indent
\usepackage{graphicx}				% Use pdf, png, jpg, or eps§ with pdflatex; use eps in DVI mode
								% TeX will automatically convert eps --> pdf in pdflatex		
\usepackage{amssymb}
\usepackage{amsmath}
\usepackage[shortlabels]{enumitem}
\usepackage{float}
\usepackage{tikz-cd}

\usepackage{amsthm}
\theoremstyle{definition}
\newtheorem{definition}{Definition}[section]
\newtheorem{theorem}{Theorem}[section]
\newtheorem{corollary}{Corollary}[theorem]
\newtheorem{lemma}[theorem]{Lemma}

\newcommand{\N}{\mathbb{N}}
\newcommand{\R}{\mathbb{R}}
\newcommand{\Z}{\mathbb{Z}}
\newcommand{\Q}{\mathbb{Q}}

\usepackage{simpler-wick}

% make arrow superscripts
\DeclareFontFamily{OMS}{oasy}{\skewchar\font48 }
\DeclareFontShape{OMS}{oasy}{m}{n}{%
         <-5.5> oasy5     <5.5-6.5> oasy6
      <6.5-7.5> oasy7     <7.5-8.5> oasy8
      <8.5-9.5> oasy9     <9.5->  oasy10
      }{}
\DeclareFontShape{OMS}{oasy}{b}{n}{%
       <-6> oabsy5
      <6-8> oabsy7
      <8->  oabsy10
      }{}
\DeclareSymbolFont{oasy}{OMS}{oasy}{m}{n}
\SetSymbolFont{oasy}{bold}{OMS}{oasy}{b}{n}

\DeclareMathSymbol{\smallleftarrow}     {\mathrel}{oasy}{"20}
\DeclareMathSymbol{\smallrightarrow}    {\mathrel}{oasy}{"21}
\DeclareMathSymbol{\smallleftrightarrow}{\mathrel}{oasy}{"24}
%\newcommand{\cev}[1]{\reflectbox{\ensuremath{\vec{\reflectbox{\ensuremath{#1}}}}}}
\newcommand{\vecc}[1]{\overset{\scriptscriptstyle\smallrightarrow}{#1}}
\newcommand{\cev}[1]{\overset{\scriptscriptstyle\smallleftarrow}{#1}}
\newcommand{\cevvec}[1]{\overset{\scriptscriptstyle\smallleftrightarrow}{#1}}

%SetFonts


\title{Renormalization}
\author{Patrick Oare}
\date{}							% Activate to display a given date or no date

\begin{document}
\maketitle
\section{Regularization-Independent Momentum Subtraction (RI-MOM)}

The RI-MOM scheme (also known as Rome-Southampton) is a renormalization scheme well equipped to deal with 
the lattice, namely because we can calculate the relevant quantities that define the scheme easily on the lattice. The 
lattice spacing $a$ provides us a with a natural UV regulator, and RI-MOM will tell us how to go from such a regulated 
result to a physical observable quantity. 

The RI-MOM has a relatively simple renormalization condition. We will define it here for an arbitrary Green's function, 
for example a three point function. For a renormalization scale $\mu$ and working in a fixed gauge, \textbf{we 
define the amputated, renormalized Green's function at momentum $p^2 = -\mu^2$ to be equal to its tree level 
value}. 

In practice on the lattice, there are two things that we must compute. We will work with a specific example here for a 
given quark field $q(x)$. Suppose the operator we are trying to compute is $\mathcal O(z)$. We will renormalize the 
three point function:
\begin{equation}
	G(p) = \frac{1}{V}\sum_{x, y, z}e^{-ip\cdot (x - y)}\langle q(x)\mathcal O(z) \overline{q}(y)\rangle
\end{equation}
i.e. we are projecting the source and sink to a definite momentum and projecting the operator $\mathcal O(z)$ to zero 
momentum. 

We also will need to compute the momentum projected propagator:
\begin{equation}
	S(p) = \frac{1}{V}\sum_{x, y}e^{-ip\cdot(x - y)} S(x, y)
\end{equation}
where $S^{ab}_{ij}(x, y) = \langle q^a_i(x)\bar q^b_j(y)\rangle$ is the standard propagator. On the lattice, the two objects that we must compute 
directly are $S(p)$ and $G(p)$, and everything else follows once we have these quantities. 

To explain the method, our goal is to compute the operator renormalization:
\begin{equation}
	\mathcal O_R(\mu) = \mathcal Z(\mu)\mathcal O_{lat}
\end{equation}
where $\mathcal O_R(\mu)$ is our renormalized operator, $\mathcal Z(\mu)$ is the renormalization coefficient of 
interest, and $\mathcal O_{lat}$ is the lattice (bare) operator. We will also assume the quark fields have been 
renormalized by some quark field renormalization $\mathcal Z_q$:
\begin{equation}
	q_{lat} = \sqrt{\mathcal Z_q} q_R(\mu)
\end{equation}
There is an analytical expression for $\mathcal{Z}_q$ on the lattice in the RI-MOM scheme, and it has been determined 
to be:
\begin{equation}
	\mathcal Z_q(p)|_{p^2 = -\mu_R^2} = \left[\frac{tr\left\{-i\sum_{\nu = 1}^4 \gamma_\nu \sin(ap_\nu) a S(p)^{-1}\right\}}{12\sum_{\nu = 1}^4 \sin^2(ap_\nu)}\right]_{p^2 = -\mu^2}
\end{equation}
The twelve on the bottom is a normalization $12 = 3\times 4$ for the number of color and number of spin indices, which we will 
see in many of the expressions. 

Let $\Gamma(p)$ be the amputated three point function, bare or renormalized. We can relate $\Gamma$ to the other 
quantities we have already computed by using the inverse of the propagator to manually cut the legs off the full three 
point function:
\begin{equation}
	\Gamma(p) = S(p)^{-1} G(p) S(p)^{-1}
\end{equation}
We will denote the tree level version of this by $\Gamma_B$, where the $B$ subscript stands for Born term. 

Using our quantities already computed on the lattice, we can compute the bare $\Gamma_{lat}(p)$ directly in terms 
of the renormalized quantities. In the continuum limit, the renormalized Green's function will be:
\begin{align}
	G_R(p; \mu) &= \int d^4x \, d^4 y\, d^4 z\, e^{-ip\cdot (x - y)} \langle q_R(x; \mu) \mathcal O_R(z; \mu) q_R(y; \mu) 
	\rangle \\
	&= \mathcal Z_q(\mu)^{-1} \mathcal Z(\mu) \int d^4x \, d^4 y\, d^4 z\, e^{-ip\cdot (x - y)} \langle q_{lat}(x) \mathcal O_{lat}
	(z) q_{lat}(y) \rangle \\
	&= \mathcal Z_q(\mu)^{-1} \mathcal Z(\mu) G_{lat}(p)
\end{align}
This relation translates immediately to the amputated Green's function $\Gamma(p)$:
\begin{equation}
	\Gamma_R(p; \mu) = \mathcal Z_q(\mu)^{-1}\mathcal Z(\mu) \Gamma_{lat}(p)
\end{equation}

We are now in a position to apply the renormalization condition. We must equate the renormalized, amputated Green's function 
to the tree level Green's function $\Gamma_B(p)$. This gives us:
\begin{equation}
	\mathcal Z_q(\mu)^{-1}\mathcal Z(\mu) \Gamma(p) = \Gamma_B(p)
\end{equation}
Dividing by a conventional factor of 12 as a normalization and inverting, we can clean this expression up into a simple equation 
for $\mathcal Z(\mu)$:
\begin{equation}
	\mathcal Z(p^2 = -\mu^2) = \left[\frac{12\mathcal Z_q(p)}{tr\left\{\Gamma(p)\Gamma_B(p)^{-1}\right\}}\right]_{p^2 
	= -\mu^2}
\end{equation}

\section{Continuum Schemes: $\overline{MS}$, on-shell, and off-shell}

\section{Matching between schemes}

\section{Example: Isospin}

We are interested here in computing matrix elements of the operator:
\begin{equation}
	\mathcal O(z) = \mathcal O_u(z) - \mathcal O_d(z)~
	\label{eq:operator_dfn}
\end{equation}
where the quark operators $\mathcal O_q$ are given by
\begin{equation}
	\mathcal O_q(z) = \frac{1}{\sqrt{2}}(\mathcal T^q_{33}(z) - \mathcal T^q_{44}(z))
\end{equation}
and the irreducible tensor operators $\mathcal T^q_{\mu\nu}$ are defined as
\begin{equation}
	\mathcal T^q_{\mu\nu} = \overline q(z)\, \gamma_{\{\mu} \cevvec{D}_{\nu\}}\, q(z)
\end{equation}
with $\cevvec D = \vec D - \cev D$ the symmetrized covariant derivative. Note we define the symmetric and traceless 
component of a tensor to be:
\begin{equation}
	a_{\{\mu}b_{\nu\}} = \frac{1}{2}(a_\mu b_\nu + a_\nu b_\mu) - \frac{1}{4}a_\alpha b^\alpha g_{\mu\nu}
\end{equation} 

We are inserting the operator $\mathcal O_q$ with momentum $\vec p = 0$. We will focus our analysis on the tensor operator 
$\mathcal T^q_{\mu\mu}$ (note there is no sum on $\mu$ here), and note that $\mathcal O(z)$ can be obtained through 
linearity. We may write:
\begin{equation}
	\sum_z\mathcal T^q_{\mu\mu}(z) = \sum_{z, z'}\overline q(z)\, J_\mu(z, z')\,q(z')~
	\label{eq:operator_mom_proj}
\end{equation}
Plugging in the definition of the derivatives:
\begin{align}
	\vec D\psi(z) &= \frac{1}{2}\left(U_\mu(z)\psi(z + \hat\mu) - U_\mu(n - \hat\mu)^\dagger \psi(z - \hat\mu)\right) \\
	\overline\psi(z) \cev D &= \frac{1}{2}\left(\overline\psi(z + \hat\mu)U_\mu(z)^\dagger - 
	\overline\psi(z - \hat\mu) U_\mu(z - \hat\mu)\right)
\end{align}
we find the current $J_\mu(z, z')$ is:
\begin{equation}
	J_\mu(z, z') = \left[U_\mu(z) \delta_{z + \hat\mu, z'} - U_\mu(z')^\dagger\delta_{z - \hat\mu, z'}\right]\gamma_\mu
\end{equation}

We may now use this expansion to compute the three point function for the operator $\mathcal T_\mu = 
\mathcal T^u_{\mu\mu} - \mathcal T^d_{\mu\mu}$ (we can simply take $\mathcal T_3 - \mathcal T_4$ to get the operator of 
interest in Equation~\ref{eq:operator_dfn}). Using Equation~\ref{eq:operator_mom_proj}, we write 
\begin{equation}
	\sum_z T_\mu(z) = \sum_{z, z'}\left[\overline u(z)\, J_\mu(z, z')\, u(z') - \overline d(z)\, J_\mu(z, z')\, d(z')\right]
\end{equation}
Plugging this in, we find that for the up quark Green's function (here $\alpha, \beta$ are 
Dirac indices):
\begin{align}
	G_\mu^{\alpha\beta}(p) &= \sum_{x, y, z} e^{-ip(x - y)}\langle u^\alpha(x)\mathcal T_\mu(z) \overline u^\beta(y)\rangle \\
	&= \sum_{x, y, z, z'} e^{-ip(x - y)}\left[\langle u^\alpha(x)\overline u^\sigma(z) J_\mu^{\sigma\rho}(z, z') u(z')^\rho\overline 
	u^\beta(y)\rangle - \langle u^\alpha(x)\overline d^\sigma(z) J_\mu^{\sigma\rho}(z, z') d^\rho(z') \overline u^\beta(y)\rangle 
	\right]
\end{align}
Now we perform all possible Wick contractions on the matrix elements to write them as propagators:
\begin{align}
	\langle u^\alpha(x)\overline u^\sigma(z) &J_\mu^{\sigma\rho}(z, z') u^\rho(z')\overline u^\beta(y)\rangle = 
	\langle \wick{
		\c1 u \c1{\overline u} J \c2 u \c2{\overline u}
	} \rangle + \langle \wick{
		\c1 u \c2{\overline u} J \c2 u \c1{\overline u} 
	} \rangle \nonumber \\ 
	&= S^{\alpha\sigma}(x, z) J^{\sigma\rho}_\mu(z, z') S^{\rho\beta}(z', y) + (-1)^3 S^{\alpha\beta}(x, y) J^{\sigma\rho}_\mu(z, z')
	S^{\rho\sigma}(z', z) \\
	\nonumber\\
	\langle u^\alpha(x)\overline d^\sigma(z) &J_\mu^{\sigma\rho}(z, z') d^\rho(z') \overline u^\beta(y)\rangle = 
	\langle \wick{
		\c1 u \c2{\overline d} J \c2 d \c1{\overline u}
	}\rangle\nonumber \\ 
	&= (-1)^3 S^{\alpha\beta}(x, y) J^{\sigma\rho}_\mu(z, z')S^{\rho\sigma}(z', z)
\end{align}
where the factors of $(-1)$ come from rearranging the contraction so that the contracted pieces are of the form $\langle u\overline 
u\rangle$. The vacuum pieces cancel because the up and down quark propagators are degenerate, so the final result is very clean:
\begin{equation}
	G_\mu(p) = \sum_{x, y, z, z'}e^{-ip(x - y)} S(x, z) J(z, z') S(z', y)
\end{equation}


\end{document}